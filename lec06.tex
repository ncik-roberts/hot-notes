\sectionwithdate{Algorithmic Rules for the SKC}{2/1/2018}

\subsection{Judgments}
These should look familiar. We update them to account for singleton kinds.

\begin{align*}
  \Gamma^+ &\vdash e^+ \Rightarrow \tau^- &&\text{type synthesis/inference}\\
  \Gamma^+ &\vdash e^+ \Leftarrow \tau^+ &&\text{type checking/analysis}\\
  \Gamma^+ &\vdash c^+ \Rightarrow \kappa^- &&\text{kind synthesis}\\
  \Gamma^+ &\vdash c^+ \Leftarrow \kappa^+ &&\text{checking kind}\\
  \Gamma^+ &\vdash \kappa^+ \Leftarrow \kind &&\text{kind checking (a kind)}\\
  \Gamma^+ &\vdash \kappa^+ \trianglelefteq \kappa'^+ &&\text{subkinding}\\
  \Gamma^+ &\vdash c^+ \Leftrightarrow c'^+ &&\text{algorithmic equivalence}\\
  \Gamma^+ &\vdash q_1^+ \leftrightarrow q_2^+ : \kappa^- &&\text{algorithmic path equivalence}\\
  \Gamma^+ &\vdash c^+ \Downarrow q^- &&\text{weak-head normalization}\\
  \Gamma^+ &\vdash c^+ \leadsto c^- &&\text{weak-head reduction}\\
  \Gamma^+ &\vdash p^+ \uparrow \kappa^- &&\text{natural kind}
\end{align*}

The main thing to note is that we now need a context for weak-head reduction, since
it tells us which variables are bound to which singletons.

\subsection{Terms}
For type checking, we import \thref{2:ty-check} directly with no change.

For type inference, we make superficial changes to \thref{2:ty-synth}. Here's a representative
updated rule, with a context for weak-head normalization:

\[ \infer{\Gamma \vdash e_1~e_2 \Rightarrow \tau'}
        {\Gamma \vdash e_1 \Rightarrow \tau_1
        &\Gamma \vdash \tau_1 \Downarrow \tau \rightarrow \tau'
        &\Gamma \vdash e_2 \Leftarrow \tau
        }
\]

We also check that the kind is indeed virtuous in the type abstraction term.
\[
  \infer{\Gamma \vdash \Lambda(\alpha : \kappa).e \Rightarrow \forall(\alpha : \kappa).\tau}
    {\Gamma \vdash \kappa \Leftarrow \kind
    &\Gamma, \alpha : \kappa \vdash e \Rightarrow \tau
   }
\]

\subsection{Kinds and types}
\begin{judgment}[Checking kind]
  $\Gamma \vdash \kappa \Leftarrow \kind$
  \[
    \infer{\Gamma \vdash \T \Leftarrow \kind}{}
    \qquad
    \infer{\Gamma \vdash \S(c) \Leftarrow \kind}
          {\Gamma \vdash c \Leftarrow \T}
  \]
  \[
    \infer{
      \deduce
        {\Gamma \vdash \Pi(\alpha : \kappa_1).\kappa_2 \Leftarrow \kind}
        {\Gamma \vdash \Sigma(\alpha : \kappa_1).\kappa_2 \Leftarrow \kind}
      }
      {\Gamma \vdash \kappa_1 \Leftarrow \kind
      &\Gamma, \alpha : \kappa_1 \Leftarrow\kind
      }
  \]
\end{judgment}

\begin{judgment}[Kind checking]
  $\Gamma \vdash c \Leftarrow \kappa$
  \[
    \infer
      {\Gamma \vdash c \Leftarrow \kappa}
      {\Gamma \vdash c \Rightarrow \kappa'
      &\Gamma \vdash \kappa' \trianglelefteq \kappa
      }
  \]
\end{judgment}

\begin{judgment}[Kind synthesis]
  $\Gamma \vdash c \Rightarrow \kappa$
  \[
    \infer[\star]
      {\Gamma \vdash \alpha \Rightarrow \S(\alpha : \kappa)}
      {\Gamma(\alpha) = \kappa}
  \]
  \[
    \infer{\Gamma \vdash \lambda(\alpha : \kappa).c \Rightarrow \Pi(\alpha : \kappa). \kappa'}
      {\Gamma, \alpha : \kappa \vdash c \Rightarrow \kappa'
      &\Gamma \vdash \kappa \Leftarrow \kind
      }
    \qquad
    \infer{\Gamma \vdash c_1~c_2 \Rightarrow [c_2 / \alpha] \kappa'}
      {\Gamma \vdash c_1 \Rightarrow \Pi(\alpha : \kappa). \kappa'
      &\Gamma \vdash c_2 \Leftarrow \kappa
      }
  \]
  \[
    \infer[\star]{\Gamma \vdash \langle c_1, c_2 \rangle : \kappa_1 \times \kappa_2}
      {\Gamma \vdash c_1 \Rightarrow \kappa_1
      &\Gamma \vdash c_2 \Rightarrow \kappa_2
      }
    \qquad
    \infer{\Gamma \vdash \pi_1c \Rightarrow \kappa_1}
      {\Gamma \vdash c \Rightarrow \Sigma(\alpha : \kappa_1).\kappa_2}
    \qquad
    \infer{\Gamma \vdash \pi_2c \Rightarrow [\pi_1c/\alpha]\kappa_2}
      {\Gamma \vdash c \Rightarrow \Sigma(\alpha : \kappa_1).\kappa_2}
  \]
  \[
    \infer{\Gamma \vdash \tau_1 \rightarrow \tau_2 \Rightarrow \S(\tau_1 \rightarrow \tau_2)}
      {\Gamma \vdash \tau_1 \Leftarrow \T
      &\Gamma \vdash \tau_2 \Leftarrow \T
      }
    \qquad
    \infer
      {\Gamma \vdash \forall (\alpha: \kappa).\tau \Rightarrow \S(\forall(\alpha : \kappa).\tau)}
      {\Gamma \vdash \kappa \Leftarrow \kind
      &\Gamma, \alpha : \kappa \vdash \tau \Leftarrow \T
      }
  \]
\end{judgment}

The singleton rule is where the action is. We don't just report the kind of $\alpha$ as
$\T$; we want to report the \emph{principal} kind of $\alpha$.

In the application rule, notice that we don't weak-head normalize the kind inferred for
$c_1$. This is because there's only one way for a dependent product kind to look. (The version
of this statement for $\T$ doesn't hold, of course, because of singleton kinds.)

For the introductory form for product, you might be alarmed that we're not using a dependent sum.
Don't be---there's always a way to write the kind of a pair without the dependent sum. However,
when dealing with, say, a variable given a dependent sum kind, we still need the machinery to
deal with that. But be careful to correctly shift de Bruijn indices with this rule.

\begin{judgment}[Subkinding]
  $\Gamma \vdash \kappa_1 \trianglelefteq \kappa_2$
  \[
    \infer{\Gamma \vdash \T \trianglelefteq \T}{}
    \qquad
    \infer[\star]{\Gamma \vdash \S(c) \trianglelefteq \T}{}
  \]
  \[
    \infer{\Gamma \vdash \S(c) \trianglelefteq \S(c')}{\Gamma \vdash c \Leftrightarrow c' : \T}
  \]
  \[
    \infer{\Gamma \vdash \Sigma(\alpha : \kappa_1). \kappa_2 \trianglelefteq
      \Sigma(\alpha : \kappa_1').\kappa_2'}
      {\Gamma \vdash \kappa_1 \trianglelefteq \kappa_1'
      &\Gamma, \alpha : \kappa_1 \vdash \kappa_2 \trianglelefteq \kappa_2'
      }
    \qquad
    \infer{\Gamma \vdash \Pi(\alpha : \kappa_1). \kappa_2 \trianglelefteq \Pi(\alpha: \kappa_1'). \kappa_2'}
    {\Gamma \vdash \kappa_1' \trianglelefteq \kappa_1
    &\Gamma, \alpha : \kappa_1' \vdash \kappa_2 \trianglelefteq \kappa_2'
    }
  \]
\end{judgment}

These algorithmic rules don't contain provisions for kind-goodness checks in the same way
that the declarative rules do. We only perform such checks when evaluating a kind provided by
the programmer, such as in the $\Lambda$ term. We omit these checks because it's sound to do so.
I omit the list of soundness rules because they are only interesting if you're proving properties
about these rules, which I am not.

%TODO: Stop being a baby and transcribe the soundness rules

\begin{judgment}[Algorithmic equivalence] $\Gamma \vdash c \Leftrightarrow c' : \kappa$
  \[
    \infer[\star]{\Gamma \vdash c \Leftrightarrow c' : \S(c'')}{}
  \]
  \[
    \infer[\star]
      {\Gamma \vdash c \Leftrightarrow c' : \Pi(\alpha : \kappa_1). \kappa_2}
      {\Gamma, \alpha : \kappa_1 \vdash c~\alpha \Leftrightarrow c"~\alpha : \kappa_2}
    \qquad
    \infer
      {\Gamma \vdash c \Leftrightarrow c' : \Sigma(\alpha : \kappa_1). \kappa_2}
      {\Gamma \vdash \pi_1c \Leftrightarrow \pi_1 c' : \kappa_1
      &\Gamma \vdash \pi_2c \Leftrightarrow \pi_2c' : [\pi_1c/\alpha]\kappa_2
      }
  \]
  \[
    \infer{\Gamma \vdash c_1 \Leftrightarrow c_2 : \T}
      {\Gamma \vdash c_1 \Downarrow c_1'
      &\Gamma\vdash c_2 \Downarrow c_2'
      &\Gamma \vdash c_1' \leftrightarrow c_2' : \T
      }
  \]
\end{judgment}

The singleton rule follows from regularity (i.e. soundness). The other starred rule is starred
so you know to watch out for shifting indices.

The kind has gotten smaller throughout this judgment. However, when dealing with structural
equivalence, the kind actually gets \emph{larger}, and that's why it's so dang hard to prove
termination for the algorithm described by these rules.

Weak-head normalization is the same as \thref{2:whn}, but now threading a context $\Gamma$
through the premises and conclusions.

\begin{judgment}[Weak-head reduction]
  $\Gamma \vdash c \leadsto c'$
  \[
    \infer{\Gamma \vdash (\lambda(\alpha : \kappa).c)~c' \leadsto [c'/\alpha]c}{}
    \qquad
    \infer{\Gamma \vdash c_1~c_2 \leadsto c_1'~c_2}{\Gamma \vdash c_1 \leadsto c_1'}
  \]
  \[
    \infer{\Gamma \vdash \pi_i~\langle c_1, c_2 \rangle \leadsto c_i}{}
    \qquad
    \infer{\Gamma \vdash \pi_i~c \leadsto \pi_i~c'}{\Gamma\vdash c \leadsto c'}
  \]
  \[
    \infer[\star]{\Gamma \vdash c \leadsto c'}{\Gamma \vdash c \uparrow \S(c')}
  \]
\end{judgment}

We want the starred rule to only apply to paths, since otherwise it's wasteful. You can
imagine this rule as looking up the path in the context, but as stupidly as possible. Don't
be clever.

\begin{judgment}[Natural kind. Don't be clever.]
  $\Gamma \vdash p \uparrow \kappa$
  \[
    \infer{\Gamma \vdash \alpha \uparrow \kappa}{\Gamma(\alpha) = \kappa}
    \qquad
    \infer{\Gamma \vdash p~c \uparrow [c/\alpha]\kappa_2}
      {\Gamma \vdash p \uparrow \Pi(\alpha : \kappa_1).\kappa_2}
    \qquad
    \infer
      {\deduce
        {\Gamma \vdash \pi_1p \uparrow \kappa_1}
        {\Gamma \vdash \pi_2p \uparrow [\pi_1p/\alpha]\kappa_2}
      }
      {\Gamma \vdash p \uparrow \Sigma(\alpha : \kappa_1).\kappa_2}
  \]
\end{judgment}

Remember that whnf is either a lambda, a $\forall$-type, or a path. We don't want $\uparrow$ rules
for lambda or $\forall$, since $\Gamma \vdash \lambda \cdots \uparrow \T$, which is not even
what we're looking for with the natural kind rules. There's also no need to look up
a natural kind for $c_1$ when performing the step $c_1~c_2 \leadsto c_1'~c_2$---this
constructor can never be a singleton, so why bother looking?

There is also a notion of completeness for these algorithmic rules with respect to the
declarative rules from last lecture.

\emph{Note:} Structural equivalence was not covered. We resume with that next lecture.
