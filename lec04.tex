\sectionwithdate{Singleton Kind Calculus}{1/25/2018}

\subsection{Remarks from previous lecture}
\begin{itemize}
  \item Although in the previous section I treated $M[\sigma]$ as a term, Prof\@. Crary
    wanted to present it as an operation on terms.
  \item $M \cdot \uparrow^k \equiv M \cdot k \cdot k+1 \cdot \cdots$
\end{itemize}

\subsection{Toward the Singleton Kind Calculus}
Starting with $F_\omega$ + products, let's add one more kind scheme.
\begin{bnf}
  \kappa \bnfeq \T
  \alt \kappa \to \kappa
  \alt \kappa \times \kappa
  \alt \S(c)
\end{bnf}
From a set-theoretic point of view, $\S(c)$ is the singleton set $\{ c \}$, but we of
course look down on set theory, so don't go too far with this.

For sure, $\int : \S(\int)$. But we also want that $(\lambda \alpha. \alpha)~\int : \S(\int)$.
(The kind of the parameter $\alpha$ was unspecified in class, but it seems that we want
to allow $\alpha : \T$ in addition to the obvious $\alpha : \S(\int)$.)

Let's review the components of a basic ML signature:
\begin{verbatim}
    type t
    type u
    val a : t
\end{verbatim}
The type component of this signature is $\alpha : \T \times \T$, and the value component
is $a : \pi_2~\alpha$. In an explicit, made-up ML syntax, this would look like:
\begin{verbatim}
    typeconstructor t : type
    typeconstructor u : type
    val a : t
\end{verbatim}

This is nice and breezy, but what happens if we need to introduce sharing constraints
between types? The classic example of the non-geopolitical Diamond Import Problem is where
a lexer and parser signature both reference a symbol table module.
A compiler, needing to make use of both a lexer and a parser, must specify that its lexer
and parser don't deviate in symbol table.
(A remark was made that Haskell solves this using ``sharing by construction (parameterization),''
in contrast to the ML orthodoxy's ``sharing by specification (fibration).''
There are category-theoretical implications to this all, believe it or not. I don't.)

A complication: how to kind $\lambda(\alpha: \T). \int$? Surely it's $\T \to \T$,
but also it's $\T \to \S(\int)$. Fine, but what about $\lambda(\alpha : \T) : \alpha$?
Again, $\T \to \T$ should groove, but we have no way to assign the kind $\T \to \S(?)$.
This formulation doesn't appear to be entirely satisfactory, so let's nix it and throw
dependent kinds into the mix.

\subsection{Dependent kinds}
We instead adopt this grammar of kinds.
\begin{bnf}
  \kappa \bnfeq \T
  \alt \Pi (\alpha : \kappa). \kappa
  \alt \Sigma (\alpha : \kappa). \kappa
  \alt \S(c)
\end{bnf}

Somehow, we want to set this up so:
\begin{align*}
  \lambda(\alpha : \T) . \alpha &: \Pi(\alpha : \T) . T\\
                &: \Pi(\alpha : \T) . \S(\alpha)
\end{align*}
where the second option is clearly the best. So good that no other type has that kind.

Arrows (exponentiation) are iterated products; pairs (products) are iterated sums.
We can still retain the old notation where there is no dependence.
Where we would say $\langle \int, \int\rangle : \T\times\T$ before, we now say
\mbox{$\langle \int, \int \rangle : \Sigma(\alpha : \T). \S(\alpha)$}, and eruditely.

For a signature with kind $\Sigma(\texttt{t} : \S(\int)) . \S(\texttt{t} \times \texttt{t})$,
look no further than
\begin{verbatim}
     type t = int
     type u = t * t
\end{verbatim}

But we impugn kinds with dependence.
It's possible for a kind to be bad, so we resort to the usual
interrogation tactics.
\begin{judgment}[Innocence of kinds (kind formation)]
  $\Gamma \vdash \kappa : \kind$
  \[
    \infer{\Gamma \vdash \T : \kind}{}
    \qquad
    \infer{\Gamma \vdash \S(\tau) : \kind}
      {\Gamma \vdash \tau : \T}
  \]
  \[
    \infer{\Gamma \vdash \Pi(\alpha:\kappa_1).\kappa_2 : \kind}
      {\Gamma \vdash \kappa_1 : \kind
      &\Gamma, \alpha : \kappa_1 \vdash \kappa_2 : \kind
      }
    \qquad
    \infer{\Gamma \vdash \Sigma(\alpha:\kappa_1).\kappa_2 : \kind}
      {\Gamma \vdash \kappa_1 : \kind
      &\Gamma, \alpha : \kappa_1 \vdash \kappa_2 : \kind
      }
  \]
\end{judgment}

\subsection{Equivalence in kind}
Before, in the judgment $\Gamma \vdash c_1 \equiv c_2 : \kappa$, neither the context
$\Gamma$ nor the kind $\kappa$ was needed. But, just as Mercury
casts capricious shadows on the surface of Mars, so too introduces the fallen
kind new pathways to deceit. Dependent, yes---dependent on sin.
\[\lambda(\alpha : \T).\alpha \stackrel{?}{\equiv} \lambda(\alpha:\T).\int\]

At $\T \to \T$, no. At $\S(\int) \to \T$, yeah. Subkinding helps:
$\S(\int) \to \T \ge \T \to \T$. The kind $\kappa$ is necessary to check
equivalence. The context is clearly necessary for dependent
kinds, and I don't feel like reproducing the example from lecture.

Here, our declarative judgments:
\begin{align*}
  \Gamma &\vdash \kappa : \kind &&\text{kind formation (done!)}\\
  \Gamma &\vdash \kappa_1 \equiv \kappa_2 : \kind &&\text{kind equivalence}\\
  \Gamma &\vdash \kappa_1 \le \kappa_2 &&\text{subkinding}\\
  \Gamma &\vdash c : \kappa &&\text{constructor formation}\\
  \Gamma &\vdash c_1 \equiv c_2 : \kappa &&\text{constructor equivalence}\\
  \hline
  \Gamma &\vdash e : c &&\text{term formation}\\
  &\vdash \Gamma~\mathsf{ok} &&\text{context ok-ness}
\end{align*}

Above the line is where the action is. Below, unchanged.
Let's briefly state regularity conditions. If $\vdash \Gamma~\mathsf{ok}$, then:
\begin{enumerate}[(i)]
  \item if $\Gamma \vdash \kappa \equiv \kappa' : \kind$, then $\Gamma \vdash \kappa : \kind$
    and $\Gamma \vdash \kappa : \kind$.
  \item if $\Gamma \vdash \kappa \le \kappa'$, then $\Gamma \vdash \kappa : \kind$
    and $\Gamma \vdash \kappa : \kind$.
  \item if $\Gamma \vdash c : \kappa$, then $\Gamma \vdash \kappa : \kind$.
  \item if $\Gamma \vdash c \equiv c' : \kappa$, then $\Gamma \vdash c : \kappa$
    and $\Gamma \vdash c' : \kappa$.
  \item if $\Gamma \vdash e : \tau$, then $\Gamma \vdash \tau : \T$.
\end{enumerate}

We start with defining kind checking for types.
\begin{judgment}[Constructor formation (incomplete)]\thlabel{sgk:type}
  $\Gamma \vdash c : \kappa$
  \[
    \infer{\Gamma \vdash \alpha : \kappa}{\Gamma(\alpha) = \kappa}
  \]
  \[
    \infer{\Gamma \vdash \lambda(\alpha : \kappa).c : \Pi(\alpha : \kappa).\kappa'}
      {\Gamma \vdash \kappa : \kind
      &\Gamma, \alpha : \kappa \vdash c : \kappa'
      }
    \qquad
    \infer{\Gamma \vdash c_1~c_2 : [c_2 / \alpha]\kappa'}
      {\Gamma \vdash c_1 : \Pi(\alpha:\kappa).\kappa'
      &\Gamma \vdash c_2 : \kappa
      }
  \]
  \[
    \infer{\Gamma \vdash \langle c_1, c_2 \rangle : \Sigma(\alpha : \kappa_1).\kappa_2}
      {\Gamma, \alpha : \kappa_1 \vdash \kappa_2 : \kind
      &\Gamma \vdash c_1 : \kappa_1
      &\Gamma \vdash c_2 : [c_1/\alpha]\kappa_2
      }
  \]
\end{judgment}
In the elimination rule for dependent products, $\kappa$ and $\kappa'$ live under a different
number of binders. Watch out when implementing de Bruijn indices.

The regularity condition in the premise of the dependent sum rule is required.
For example, if $\gamma : \S(\int) \to \T$, the substitution of a type constructor for $\gamma$
in $\langle \int, \gamma~\int\rangle$ will appear to license the kind $\Sigma(\alpha:\T).\S(\gamma~\alpha)$,
but this in fact should not be the case.

That the dependent sum is dual to the dependent product is to be expected, and is
``standard for dependent types.'' This is not entirely satisfactory, so hopefully more
exposure to this notion will make this clear. (Presumably, $\pi_1$ and $\pi_2$ are
still eliminatory for dependent sums, but this went unmentioned.)

\subsection{Remarks}
\begin{itemize}
\item Why can't the declaration \texttt{type t = int} be interpreted as introducing an
  alias for the type $\int$? Prof\@. Crary's explanation was that the signature with this
  binding is a subkind of the signature specifying only \texttt{type t}, and that the
  alias interpretation is not satisfactory in this regard.
\item For more on the Diamond Import Problem, see Pierce's \emph{Advanced Topics in Types and
  Programming Languages}.
\end{itemize}
