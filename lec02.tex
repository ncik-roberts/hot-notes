\sectionwithdate{Writing a typechecker for $F_\omega$}{1/18/2018}

In this lecture, we rephrase the declarative syntax in the language of moded
judgments so that we can consider matters of input and output. As before, since
a type may admit at most one kind in $F_\omega$, much of the material introduced
here will stay unnecessary until we have reached the singleton kind calculus.

\subsection{Judgments}

Here is a list of all judgments defined in today's lecture. We use a superscript $+$
to indicate the input and a superscript $-$ to indicate the output.
\begin{align*}
  \Gamma^+ &\vdash e^+ \Rightarrow \tau^- &&\text{type synthesis/inference}\\
  \Gamma^+ &\vdash e^+ \Leftarrow \tau^+ &&\text{type checking/analysis}\\
  \Gamma^+ &\vdash c^+ \Rightarrow \kappa^- &&\text{kind synthesis}\\
  \Gamma^+ &\vdash c^+ \Leftarrow \kappa^+ &&\text{kind checking}\\
  \Gamma^+ &\vdash c+^ \Leftrightarrow c'^+ &&\text{algorithmic equivalence}\\
  \Gamma^+ &\vdash q_1^+ \leftrightarrow q_2^+ : \kappa^- &&\text{algorithmic path equivalence}\\
  &c^+ \Downarrow q^- &&\text{weak-head normalization}\\
  &c^+ \leadsto c^- &&\text{weak-head reduction}
\end{align*}

\subsection{Inference Rules}

\begin{judgment}[Type synthesis]
\[ \infer{\Gamma \vdash x \Rightarrow \tau}{\Gamma(x) = \tau} \]
\[
  \infer
    {\Gamma \vdash \lambda (x : \tau).e \Rightarrow \tau \rightarrow \tau'}
    {\Gamma \vdash \tau \Leftarrow \T
    &\Gamma,x : \tau \vdash e \Rightarrow \tau'
    }
  \qquad
  \infer
    {\Gamma \vdash e_1~e_2 \Rightarrow \tau_2}
    {\Gamma \vdash e_1 \Rightarrow \tau
    &\tau \Downarrow \tau_1 \rightarrow \tau_2
    &\Gamma \vdash e_2 \Leftarrow \tau_1
    }
\]
\end{judgment}

Unlike 15-317, we can synthesize the type of a lambda term since we have the type
of the parameter specified in the term.
In the last rule, we don't simply synthesize the type of $e_1$ and confirm that it
is an arrow type.  This is because the type could have redexes. We instead normalize
to $\tau_1 \rightarrow \tau_2$.\footnote{I'm still not clear on why we can rely on WHNF
to correctly ``normalize'' when it seemed to be a theme of lecture that we cannot rely
on this.}

\begin{judgment}[Type checking]
\[
  \infer
    {\Gamma \vdash e \Leftarrow \tau}
    {\Gamma \vdash e \Rightarrow \tau'
    &\Gamma \vdash \tau \Leftrightarrow \tau' : \T
    }
  \qquad
  \infer
    {\Gamma \vdash \Lambda (\alpha : \kappa).e \Rightarrow \forall (\alpha : \kappa).\tau}
    {\Gamma, \alpha : \kappa \vdash e \Rightarrow \tau}
\]
\[
  \infer
    {\Gamma \vdash e[c] \Rightarrow [c/\alpha]\tau'}
    {\Gamma \vdash e \Rightarrow \tau
    &\tau \Downarrow \forall (\alpha : \kappa).\tau'
    &\Gamma \vdash c \Leftarrow \kappa
    }
\]
\end{judgment}

We do not need to check the validity of kinds---\emph{yet} (ominously). The same footnote
holds for the universal type.

\paragraph{Algorithmic equivalence.} An option that works for $F_\omega$ is to repeatedly
contract redeces until reaching a normalized term. However, this doesn't generalize
to the singleton kind calculus, so we develop more machinery here.

\begin{judgment}[Kind synthesis]
  \[ \infer{\Gamma \vdash \alpha \Rightarrow \kappa}{\Gamma(\alpha) = \kappa} \]
  \[\infer{\Gamma \vdash \lambda (\alpha : \kappa).c \Rightarrow \kappa \rightarrow \kappa'}
         {\Gamma, \alpha : \kappa \vdash c \Rightarrow \kappa'}
   \qquad
   \infer{\Gamma \vdash c_1~c_2 \Rightarrow \kappa'}
         {\Gamma \vdash c_1 \Rightarrow \kappa \rightarrow \kappa'
         &\Gamma \vdash c_2 \Leftarrow \kappa
         }
  \]
  \[
    \infer{\Gamma \vdash \langle c_1, c_2 \rangle \Rightarrow \kappa_1 \times \kappa_2}
      {\Gamma \vdash c_1 \Rightarrow \kappa_1
      &\Gamma \vdash c_2 \Rightarrow \kappa_2
      }
    \qquad
    \infer{\Gamma \vdash \pi_i~c \Rightarrow \kappa_i}
      {\Gamma \vdash c \Rightarrow \kappa'
      &\Gamma \vdash c \Rightarrow \kappa_1 \times \kappa_2
      }
  \]
  \[
    \infer{\Gamma \vdash c_1 \rightarrow c_2 \Rightarrow \T}
      {\Gamma \vdash c_1 \Leftarrow \T
      &\Gamma \vdash c_2 \Leftarrow \T
      }
    \qquad
    \infer{\Gamma \vdash \forall(\alpha : \kappa).\tau \Rightarrow \T}
      {\Gamma, \alpha : \kappa \vdash \tau \Leftarrow \T}
  \]
  \[
    \infer[\text{(check)}]
      {\Gamma \vdash c \Leftarrow \kappa}
      {\Gamma \vdash c \Rightarrow \kappa'
      &\kappa = \kappa'
      }
  \]
\end{judgment}

We write $\emph{(check)}$ in such a way that we can more explicitly see what we'll
generalize when we move to other calculi, particularly the notion of equality.

Now when we write the equivalence rules, think about extensionality. We recurse on the
\emph{kind} using the appropriate projection or function application. What we \emph{don't}
do is normalize both types and check equivalence, because this doesn't generalize to
the singleton kind calculus.

\begin{judgment}[Algorithmic equivalence]\thlabel{algequiv}
\[ \infer
    {\Gamma \vdash c \Leftrightarrow c' : \kappa_1 \rightarrow \kappa_2}
    {\Gamma, \alpha : \kappa_1 \vdash c~\alpha \Leftrightarrow c'~\alpha : \kappa_2}
  \qquad
  \infer
    {\Gamma \vdash c \Leftrightarrow c' : \kappa_1 \times \kappa_2}
    {\Gamma \vdash \pi_1~c \Leftrightarrow \pi_1~c' : \kappa_1
    &\Gamma \vdash \pi_2~c \Leftrightarrow \pi_2~c' : \kappa_2
    }
\]
\[
  \infer
    {\Gamma \vdash c \Leftrightarrow c' : \T}
    {c \Downarrow q
    &c' \Downarrow q'
    &\Gamma \vdash q \leftrightarrow q' : \T
    }
\]
\end{judgment}
We can only normalize once we reach a type kind.

A normal form is such that all redeces have been contracted. However, we
only place in \emph{weak head} normal form, where an arrow constructor or
a universal constructor is at the outermost level. (That is, we don't recursively
normalize.)

\begin{judgment}[Weak-head normalization]
\[
  \infer
    {c \Downarrow c''}
    {c \leadsto c' & c' \Downarrow c''}
  \qquad
  \infer
    {c \Downarrow c}
    {c \not\leadsto}
\]
\end{judgment}
Don't be too concerned about $c \not\leadsto$. In practice, we could implement
this in ML by raising an exception if no reduction step is made.

\begin{judgment}[Weak-head reduction]\thlabel{weak}
\[
  \infer{(\lambda(\alpha : \kappa).c)~c' \leadsto [c'/\alpha]c}{}
  \qquad
  \infer{\pi_i~\langle c_1, c_2 \rangle \leadsto c_i}{}
\]
\[
  \infer{c_1~c_2 \leadsto c_1'~c_2}{c_1 \leadsto c_1'}
  \qquad
  \infer{\pi_i~c \leadsto \pi_i~c'}{c \leadsto c'}
\]
\end{judgment}
We only reduce under an application and a projection. This is so that when we
encounter types such as $((\lambda \alpha. \alpha)~(\lambda \alpha.\alpha))~c_3$,
we can reduce them.

The definitions for path and whnf are curiously familiar.
\begin{bnf}
  \text{path}~p \bnfeq
  \alpha
  \alt p~c
  \alt \pi_1~p
  \alt \pi_2~p\\
  \text{whnf}~q \bnfeq
  p
  \alt c \rightarrow c
  \alt \forall (\alpha : \kappa).c
\end{bnf}
Path is clearly a neutral term. Whnf is almost a neutral term, except
the constituents (body of universal type and right/left of arrow may contain
redeces). We have already gotten to kind $\T$, which guarantees that
this grammar is exhaustive.

Algorithmic structural equivalence should look familiar from constructive logic.
The kind is synthesized as an output. Sometimes it is put back in as an input to
the algorithmic equivalence judgment.
\begin{judgment}[Algorithmic structural equivalence]
\[
  \infer{\Gamma \vdash \alpha \leftrightarrow \alpha : \kappa}{\Gamma(\alpha) = \kappa}
  \qquad
  \infer
    {\Gamma \vdash : p~c \leftrightarrow p'~c' : \kappa_2}
    {\Gamma \vdash p \leftrightarrow p' : \kappa_1 \rightarrow \kappa_2
    &\Gamma \vdash c \Leftrightarrow c' : \kappa_1
    }
\]
\[
  \infer
    {\Gamma \vdash \pi_i~p \leftrightarrow \pi_i~p' : \kappa_i}
    {\Gamma \vdash p \leftrightarrow p' : \kappa_1 \times \kappa_2}
  \qquad
  \infer
    {\Gamma \vdash (c_1 \rightarrow c_2) \leftrightarrow (c_1' \rightarrow c_2') : \T}
    {\Gamma \vdash c_1 \Leftrightarrow c_1' : \T
    &\Gamma \vdash c_2 \Leftrightarrow c_2' : \T
    }
\]
\[
  \infer{\Gamma \vdash \forall(\alpha : \kappa).c \leftrightarrow
    \forall(\alpha:\kappa).c' : \T}
    {\Gamma, \alpha : \kappa \vdash c \Leftrightarrow c' : \T}
\]
\end{judgment}

\subsection{Introduction to de Bruijn indices}
With explicit variables, you must be careful to avoid capture. So let's instead
count the number of binders. Substitution coincides with shifting of the indices.
\begin{align*}
  &\uparrow^j_i &&\text{Add $j$ to all variables, except those bound within $i$ binders}\\
  &\uparrow^j_i(\lambda.e) = \lambda.\uparrow^j_{i+1} e\\
  &\uparrow^j_i k = k &&\text{if $k < i$}\\
  &\uparrow^j_i k = k+j &&\text{otherwise}
\end{align*}
