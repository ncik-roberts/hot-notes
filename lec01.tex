\sectionwithdate{Definition of $F_\omega$}{1/16/2018}

\emph{Note:} In this section we skip the na\"ive formulation given by Prof. Crary
(without kinds) and immediately introduce kinds.

First, we define type constructors. We often refer to these simply as ``constructors.''
By convention, $\tau$ denotes a nullary constructor, but $c$ may be used in general
without fear.
\begin{bnf}
  c, \tau \bnfeq
  \alpha
  \alt c \rightarrow c
  \alt \forall(\alpha : \kappa). c
  \alt \lambda(\alpha : \kappa). c
  \alt c~c
\end{bnf}

$\alpha$ denotes a type variable. We use these in quantification and type abstraction.
It may be instantiated during application. $\kappa$ denotes a kind, which we now
define:
\begin{bnf}
  \kappa \bnfeq
  \mathtt{type}
  \alt \kappa \rightarrow \kappa
\end{bnf}
Henceforth we use $\T$ to denote \texttt{type}.
Where would we be without terms to inhabit types?
\begin{bnf}
  e \bnfeq x
  \alt \lambda(x : \tau). e
  \alt e ~ e
  \alt \Lambda(\alpha : \kappa). e
  \alt e[\tau]
\end{bnf}

Our context, $\Gamma$, may contain judgments pertaining to types and terms.
\begin{bnf}
  \Gamma \bnfeq \varepsilon
  \alt \Gamma, x : \tau
  \alt \Gamma, \alpha : \kappa
\end{bnf}

Sometimes, for the latter judgment, you will see $\alpha :: \kappa$, but this is
not too important. Proceeding from this context, we first define inductively
the judgment that $\Gamma \vdash c : \kappa$

\begin{judgment}[Type kind]\thlabel{kind}
  $\Gamma \vdash c : \kappa$
\[
  \infer{\Gamma \vdash \alpha : \kappa}{\Gamma(\alpha) : \kappa}
  \qquad
  \infer{\Gamma \vdash c_1 \rightarrow c_2 : \T}
        {\Gamma \vdash c_1 : \T
        &\Gamma \vdash c_2 : \T
        }
  \qquad
  \infer{\Gamma \vdash \forall(\alpha : \kappa).c : \T}
        {\Gamma, \alpha : \kappa \vdash c : \T}
\]
\[
  \infer{\Gamma \vdash c_1 ~ c_2 : \kappa'}
        {\Gamma \vdash c_1 : \kappa \rightarrow \kappa'
        &\Gamma \vdash c_2 : \kappa
        }
  \qquad
  \infer{\Gamma \vdash \lambda(\alpha : \kappa).c : \kappa \to \kappa'}
        {\Gamma, \alpha : \kappa \vdash c : \kappa'}
\]
\end{judgment}

Using this judgment, we next define the judgment $\Gamma \vdash e : \tau$.

\begin{judgment}[Term type]\thlabel{term}
$\Gamma \vdash e : \tau$
\[
  \infer{\Gamma \vdash x : \tau}{\Gamma(x) = \tau}
  \qquad
  \infer{\Gamma \vdash \lambda(x : \tau).e : \tau \rightarrow \tau'}
        {\Gamma, x : \tau \vdash e : \tau'
        &\Gamma \vdash \tau : \kappa
        }
  \qquad
  \infer{\Gamma \vdash e_1 ~ e_2 : \tau'}
        {\Gamma \vdash e_1 : \tau \rightarrow \tau'
        &\Gamma \vdash e_2 : \tau
        }
\]
\[
  \infer{\Gamma \vdash \Lambda (\alpha : \kappa).e : \forall (\alpha : \kappa).\tau}
        {\Gamma, \alpha : \kappa \vdash e : \tau}
  \qquad
  \infer{\Gamma \vdash e[\tau] : [\tau / \alpha]\tau'}
        {\Gamma \vdash e : \forall(\alpha : \kappa).\tau'
        &\Gamma \vdash \tau : \kappa
        }
\]
\end{judgment}

But these rules aren't sufficient: if
$f : \forall(\alpha : \T \rightarrow \T). \alpha~\int \rightarrow \unit$,
we would want \mbox{$f[\lambda(\beta : \T).\beta]~12 : \unit$}, but
currently this is not the case. We need a way to show
\mbox{$(\lambda \beta. \beta) \int \equiv \int$}. To do so,
we define a new judgment:
\[\Gamma \vdash c \equiv c' : \kappa\]
and add to \thref{term} the inference rule
\[\infer[(equivalence)]{\Gamma \vdash e : \tau'}
  {\Gamma \vdash e : \tau
  &\Gamma \vdash \tau \equiv \tau' : \T
  }
\]

\begin{judgment}[Equivalence]\thlabel{equiv}
$\Gamma \vdash c \equiv c' : \kappa$

Equivalence rules
\[
  \infer{\Gamma \vdash c \equiv c : \kappa}{\Gamma \vdash c : \kappa}
  \qquad
  \infer{\Gamma \vdash c' \equiv c : \kappa}{\Gamma \vdash c \equiv c' : \kappa}
  \qquad
  \infer{\Gamma \vdash c_1 \equiv c_3 : \kappa}
        {\Gamma \vdash c_1 \equiv c_2 : \kappa
        &\Gamma \vdash c_2 \equiv c_3 : \kappa
        }
\]

Compatibility rules
\[
  \infer{\Gamma \vdash (c_1 \rightarrow c_2) \equiv (c_1' \rightarrow c_2') : \T}
        {\Gamma \vdash c_1 \equiv c_1' : \T
        &\Gamma \vdash c_2 \equiv c_2' : \T
        }
  \qquad
  \infer{\Gamma \vdash \forall(\alpha : \kappa).c \equiv \forall(\alpha : \kappa).c' : \T}
        {\Gamma, \alpha : \kappa \vdash c \equiv c' : \T}
\]
\[
  \infer{\Gamma \vdash \lambda(\alpha : \kappa).c \equiv
          \lambda (\alpha : \kappa.c') \equiv \kappa \rightarrow \kappa'}
       {\Gamma, \alpha : \kappa \vdash c \equiv c' : \kappa'}
  \qquad
  \infer{\Gamma \vdash c_1 ~ c_2 \vdash c_1'~c_2' : \kappa'}
        {\Gamma \vdash c_1 \equiv c_1' : \kappa \rightarrow \kappa'
        &\Gamma \vdash c_2 \equiv c_2' : \kappa
        }
\]
\end{judgment}
These rules defined a congruence relation. We haven't yet added the interesting rules.
Note the enforcement that some constructors be types ($\T$), since they certainly
may not be $\lambda$-abstractions. At this point in the semester, the $\kappa$ constraints
may be omitted, since they are uniquely determined by the \textbf{regularity conditions}
(assuming $\vdash \Gamma~\mathsf{ok}$):
\begin{itemize}
  \item If $\Gamma \vdash c_1 \equiv c_2 : \kappa$, then $\Gamma \vdash c_1 : \kappa$
    and $\Gamma \vdash c_2 : \kappa$.
  \item If $\Gamma \vdash e : \tau$, then $\Gamma \vdash \tau : \T$.
\end{itemize}

Now for a more interesting rules to add to \thref{equiv}:
\[
  \infer[(\beta)]
    {\Gamma \vdash (\lambda (\alpha : \kappa). c') c \equiv [c / \alpha]c' : \kappa'}
    {\Gamma \vdash c : \kappa
    & \Gamma, \alpha : \kappa \vdash c' : \kappa'
    }
\]
We can prove:
\[
  \infer
    {(\lambda \beta. \beta)~\int \rightarrow \unit \equiv \int \rightarrow \unit : \T}
    {\infer
       {(\lambda \beta. \beta)~\int \equiv \int : \T}
       {\infer{\beta : \T \vdash \beta : \T}{}
       &\infer{\int : \T}{}
       }
    }
\]

What about $\eta$-expansion? something along the lines of:
\[
  \infer[(\eta)]
    {\Gamma \vdash c \equiv \lambda (\alpha : \kappa). c~\alpha : \kappa \rightarrow \kappa'}
    {\Gamma \vdash c : \kappa \rightarrow \kappa'}
\]

We want something more general, akin to running an experiment on two constructors of
function kind. (Adding to \thref{equiv}.)
\[
  \infer[(extensionality)]
    {\Gamma \vdash c_1 \equiv c_2 : \kappa \rightarrow \kappa'}
    {\Gamma, \alpha : \kappa \vdash c_1~\alpha \equiv c_2~\alpha : \kappa'}
\]

Exercise: prove $\eta$ from this rule.

\subsection{$F_\omega$ plus products}
\begin{bnf}
  \kappa \bnfeq
    \cdots \alt \kappa \times \kappa\\
  c \bnfeq \cdots \alt \langle c, c \rangle
  \alt \pi_1~c
  \alt \pi_2~c
\end{bnf}

Adding to \thref{kind}:
\[
  \infer{\Gamma \vdash \langle c_1, c_2 \rangle : \kappa_1 \times \kappa_2}
    {\Gamma \vdash c_1 : \kappa_1
    &\Gamma \vdash c_2 : \kappa_2
    }
  \qquad
  \infer{\Gamma \vdash \pi_i~c:\kappa_i}
    {\Gamma \vdash c : \kappa_1 \times \kappa_2}
\]

Adding to \thref{equiv}, the compatibility rules:
\[
  \infer{\Gamma \vdash \langle c_1, c_2 \rangle \equiv \langle c_1', c_2' \rangle
    : \kappa_1 \times \kappa_2}
    {\Gamma \vdash c_1 \equiv c_1' : \kappa' & \Gamma \vdash c_2 \equiv c_2' : \kappa_2}
  \qquad
 \infer{\Gamma \vdash \pi_i~c_1 \equiv \pi_i~c_2 : \kappa_i}
   {\Gamma \vdash c_1 \equiv c_2 : \kappa_1 \times \kappa_2}
\]

Adding to \thref{equiv}, the ``interesting'' rules:
\[
  \infer[(\beta_\pi)]
    {\Gamma \vdash \pi_1 \langle c_1, c_2 \rangle \equiv c_1 : \kappa_i}
    {\Gamma \vdash c_1 : \kappa_1 & \Gamma \vdash c_2 : \kappa_2}
  \qquad
  \infer[(extensionality_\pi)]
    {\Gamma \vdash c_1 \equiv c_2 : \kappa_1 \times \kappa_2}
    {\Gamma \vdash \pi_i~c_1 \equiv \pi_i~c_2 : \kappa_i}
\]

\subsection{Motivation}
All of this is useful in the understanding of ML's module system. For example, we wish to
view the type components of a module as a singular type componen\emph{ent}, for which we
must understand a pair of types. Furthermore, functors may be seen as creating type constructors,
for example:

\begin{verbatim}
functor Foo (type t) = struct
  type u = t * t
end
\end{verbatim}
which may be represented as $\lambda (t:\T) \langle t \times t, \ldots \rangle$.

\subsection{Remarks}
\begin{itemize}
  \item We must specify a separate level (kinds) rather than just describing types with types;
    this is described in the
    \href{https://en.wikipedia.org/wiki/Burali-Forti_paradox}{Burali-Forti paradox}.
  \item Counting binders from the inside-out is called de Brujin indices, but counting from
    the outside-in is called de Brujin \emph{levels}, which ``no one uses.''
\end{itemize}
