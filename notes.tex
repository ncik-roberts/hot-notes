%----------------------------------------------------------------------------------------
%	PACKAGES AND OTHER DOCUMENT CONFIGURATIONS
%----------------------------------------------------------------------------------------

\documentclass[a4paper, 11pt]{article}
\usepackage[margin=1in]{geometry}
\usepackage{amsmath,amsfonts,amssymb}
\usepackage{theoremref}
\usepackage{enumerate}
\usepackage{environ}
\usepackage{hyperref}

\usepackage[protrusion=true,expansion=true]{microtype} % Better typography

\usepackage{mathpazo} % Use the Palatino font
\usepackage[T1]{fontenc} % Required for accented characters
\linespread{1.05} % Change line spacing here, Palatino benefits from a slight increase by default

\DeclareMathOperator{\id}{id}

\makeatletter

% Customize the title
\renewcommand{\maketitle}{
  \noindent{\LARGE\@title}
  \vspace{10pt} % Some vertical space between the title and author name

  \noindent{\@author} % Author name
  \\
  Last built: \@date % Date
  \vspace{5pt}% Some vertical space between the author block and abstract
}

%--------------------------
%   417-specific things
%--------------------------
\usepackage{proof}

% Convenient way to specify date for each lecture
\newcommand{\sectionwithdate}[2]{
  \section{#1 \hfill {\small \textnormal{\emph{(#2)}}}}
}

\NewEnviron{bnf}{
  \begin{align*}
    \BODY
  \end{align*}\ignorespacesafterend
}

\newcommand{\alt}{\; \mid \;}
\newcommand{\altline}{\\&\mid \;}
\newcommand{\bnfeq}{~~&::=~~}

\renewcommand{\int}{\mathtt{int}}
\newcommand{\unit}{\mathtt{unit}}
\newcommand{\T}{\mathtt{T}}

%-----------------------
%  Judgment
%-----------------------
\newtheorem{judgment}{Judgment}[section]

%----------------------------------------------------------------------------------------
%	TITLE
%----------------------------------------------------------------------------------------

\title{15-417 HOT Compilation (Spring 2018)}

\author{Scribe: Nick Roberts % Scribe
\\Professor: Karl Crary
\\{\textit{Carnegie Mellon University}}} % Institution

\date{\today} % Date

%----------------------------------------------------------------------------------------

\begin{document}
\maketitle % Print the title section

\sectionwithdate{Definition of $F_\omega$}{1/16/2018}

\emph{Note:} In this section we skip the na\"ive formulation given by Prof. Crary
(without kinds) and immediately introduce kinds.

First, we define type constructors. We often refer to these simply as ``constructors.''
By convention, $\tau$ denotes a nullary constructor, but $c$ may be used in general
without fear.
\begin{bnf}
  c, \tau \bnfeq
  \alpha
  \alt c \rightarrow c
  \alt \forall(\alpha : \kappa). c
  \alt \lambda(\alpha : \kappa). c
  \alt c~c
\end{bnf}

$\alpha$ denotes a type variable. We use these in quantification and type abstraction.
It may be instantiated during application. $\kappa$ denotes a kind, which we now
define:
\begin{bnf}
  \kappa \bnfeq
  \mathtt{type}
  \alt \kappa \rightarrow \kappa
\end{bnf}
Henceforth we use $\T$ to denote \texttt{type}.
Where would we be without terms to inhabit types?
\begin{bnf}
  e \bnfeq x
  \alt \lambda(x : \tau). e
  \alt e ~ e
  \alt \Lambda(\alpha : \kappa). e
  \alt e[\tau]
\end{bnf}

Our context, $\Gamma$, may contain judgments pertaining to types and terms.
\begin{bnf}
  \Gamma \bnfeq \varepsilon
  \alt \Gamma, x : \tau
  \alt \Gamma, \alpha : \kappa
\end{bnf}

Sometimes, for the latter judgment, you will see $\alpha :: \kappa$, but this is
not too important. Proceeding from this context, we first define inductively
the judgment that $\Gamma \vdash c : \kappa$

\begin{judgment}[Type kind]\thlabel{kind}
  $\Gamma \vdash c : \kappa$
\[
  \infer{\Gamma \vdash \alpha : \kappa}{\Gamma(\alpha) : \kappa}
  \qquad
  \infer{\Gamma \vdash c_1 \rightarrow c_2 : \T}
        {\Gamma \vdash c_1 : \T
        &\Gamma \vdash c_2 : \T
        }
  \qquad
  \infer{\Gamma \vdash \forall(\alpha : \kappa).c : \T}
        {\Gamma, \alpha : \kappa \vdash c : \T}
\]
\[
  \infer{\Gamma \vdash c_1 ~ c_2 : \kappa'}
        {\Gamma \vdash c_1 : \kappa \rightarrow \kappa'
        &\Gamma \vdash c_2 : \kappa
        }
  \qquad
  \infer{\Gamma \vdash \lambda(\alpha : \kappa).c : \kappa \to \kappa'}
        {\Gamma, \alpha : \kappa \vdash c : \kappa'}
\]
\end{judgment}

Using this judgment, we next define the judgment $\Gamma \vdash e : \tau$.

\begin{judgment}[Term type]\thlabel{term}
$\Gamma \vdash e : \tau$
\[
  \infer{\Gamma \vdash x : \tau}{\Gamma(x) = \tau}
  \qquad
  \infer{\Gamma \vdash \lambda(x : \tau).e : \tau \rightarrow \tau'}
        {\Gamma, x : \tau \vdash e : \tau'
        &\Gamma \vdash \tau : \kappa
        }
  \qquad
  \infer{\Gamma \vdash e_1 ~ e_2 : \tau'}
        {\Gamma \vdash e_1 : \tau \rightarrow \tau'
        &\Gamma \vdash e_2 : \tau
        }
\]
\[
  \infer{\Gamma \vdash \Lambda (\alpha : \kappa).e : \forall (\alpha : \kappa).\tau}
        {\Gamma, \alpha : \kappa \vdash e : \tau}
  \qquad
  \infer{\Gamma \vdash e[\tau] : [\tau / \alpha]\tau'}
        {\Gamma \vdash e : \forall(\alpha : \kappa).\tau'
        &\Gamma \vdash \tau : \kappa
        }
\]
\end{judgment}

But these rules aren't sufficient: if
$f : \forall(\alpha : \T \rightarrow \T). \alpha~\int \rightarrow \unit$,
we would want \mbox{$f[\lambda(\beta : \T).\beta]~12 : \unit$}, but
currently this is not the case. We need a way to show
\mbox{$(\lambda \beta. \beta) \int \equiv int$}. To do so,
we define a new judgment:
\[\Gamma \vdash c \equiv c' : \kappa\]
and add to \thref{term} the inference rule
\[\infer[(equivalence)]{\Gamma \vdash e : \tau'}
  {\Gamma \vdash e : \tau
  &\Gamma \vdash \tau \equiv \tau' : \T
  }
\]

\begin{judgment}[Equivalence]\thlabel{equiv}
$\Gamma \vdash c \equiv c' : \kappa$

Equivalence rules
\[
  \infer{\Gamma \vdash c \equiv c : \kappa}{\Gamma \vdash c : \kappa}
  \qquad
  \infer{\Gamma \vdash c' \equiv c : \kappa}{\Gamma \vdash c \equiv c' : \kappa}
  \qquad
  \infer{\Gamma \vdash c_1 \equiv c_3 : \kappa}
        {\Gamma \vdash c_1 \equiv c_2 : \kappa
        &\Gamma \vdash c_2 \equiv c_3 : \kappa
        }
\]

Compatibility rules
\[
  \infer{\Gamma \vdash (c_1 \rightarrow c_2) \equiv (c_1' \rightarrow c_2') : \T}
        {\Gamma \vdash c_1 \equiv c_1' : \T
        &\Gamma \vdash c_2 \equiv c_2' : \T
        }
  \qquad
  \infer{\Gamma \vdash \forall(\alpha : \kappa).c \equiv \forall(\alpha : \kappa).c' : \T}
        {\Gamma, \alpha : \kappa \vdash c \equiv c' : \T}
\]
\[
  \infer{\Gamma \vdash \lambda(\alpha : \kappa).c \equiv
          \lambda (\alpha : \kappa.c') \equiv \kappa \rightarrow \kappa'}
       {\Gamma, \alpha : \kappa \vdash c \equiv c' : \kappa'}
  \qquad
  \infer{\Gamma \vdash c_1 ~ c_2 \vdash c_1'~c_2' : \kappa'}
        {\Gamma \vdash c_1 \equiv c_1' : \kappa \rightarrow \kappa'
        &\Gamma \vdash c_2 \equiv c_2' : \kappa
        }
\]
\end{judgment}
These rules defined a congruence relation. We haven't yet added the interesting rules.
Note the enforcement that some constructors be types ($\T$), since they certainly
may not be $\lambda$-abstractions. At this point in the semester, the $\kappa$ constraints
may be omitted, since they are uniquely determined by the \textbf{regularity conditions}
(assuming $\vdash \Gamma~\mathsf{ok}$):
\begin{itemize}
  \item If $\Gamma \vdash c_1 \equiv c_2 : \kappa$, then $\Gamma \vdash c_1 : \kappa$
    and $\Gamma \vdash c_2 : \kappa$.
  \item If $\Gamma \vdash e : \tau$, then $\Gamma \vdash \tau : \T$.
\end{itemize}

Now for a more interesting rules to add to \thref{equiv}:
\[
  \infer[(\beta)]
    {\Gamma \vdash (\lambda (\alpha : \kappa). c') c \equiv [c / \alpha]c' : \kappa'}
    {\Gamma \vdash c : \kappa
    & \Gamma, \alpha : \kappa \vdash c' : \kappa'
    }
\]
We can prove:
\[
  \infer
    {(\lambda \beta. \beta)~\int \rightarrow \unit \equiv \int \rightarrow \unit : \T}
    {\infer
       {(\lambda \beta. \beta)~\int \equiv \int : \T}
       {\infer{\beta : \T \vdash \beta : \T}{}
       &\infer{\int : \T}{}
       }
    }
\]

What about $\eta$-expansion? something along the lines of:
\[
  \infer[(\eta)]
    {\Gamma \vdash c \equiv \lambda (\alpha : \kappa). c~\alpha : \kappa \rightarrow \kappa'}
    {\Gamma \vdash c : \kappa \rightarrow \kappa'}
\]

We want something more general, akin to running an experiment on two constructors of
function kind. (Adding to \thref{equiv}.)
\[
  \infer[(extensionality)]
    {\Gamma \vdash c_1 \equiv c_2 : \kappa \rightarrow \kappa'}
    {\Gamma, \alpha : \kappa \vdash c_1~\alpha \equiv c_2~\alpha : \kappa'}
\]

Exercise: prove $\eta$ from this rule.

\subsection{$F_\omega$ plus products}
\begin{bnf}
  \kappa \bnfeq
    \cdots \alt \kappa \times \kappa\\
  c \bnfeq \cdots \alt \langle c, c \rangle
  \alt \pi_1~c
  \alt \pi_2~c
\end{bnf}

Adding to \thref{kind}:
\[
  \infer{\Gamma \vdash \langle c_1, c_2 \rangle : \kappa_1 \times \kappa_2}
    {\Gamma \vdash c_1 : \kappa_1
    &\Gamma \vdash c_2 : \kappa_2
    }
  \qquad
  \infer{\Gamma \vdash \pi_i~c:\kappa_i}
    {\Gamma \vdash c : \kappa_1 \times \kappa_2}
\]

Adding to \thref{equiv}, the compatability rules:
\[
  \infer{\Gamma \vdash \langle c_1, c_2 \rangle \equiv \langle c_1', c_2' \rangle
    : \kappa_1 \times \kappa_2}
    {\Gamma \vdash c_1 \equiv c_1' : \kappa' & \Gamma \vdash c_2 \equiv c_2' : \kappa_2}
  \qquad
 \infer{\Gamma \vdash \pi_i~c_1 \equiv \pi_i~c_2 : \kappa_i}
   {\Gamma \vdash c_1 \equiv c_2 : \kappa_1 \times \kappa_2}
\]

Adding to \thref{equiv}, the ``interesting'' rules:
\[
  \infer[(\beta_\pi)]
    {\Gamma \vdash \pi_1 \langle c_1, c_2 \rangle \equiv c_1 : \kappa_i}
    {\Gamma \vdash c_1 : \kappa_1 & \Gamma \vdash c_2 : \kappa_2}
  \qquad
  \infer[(extensionality_\pi)]
    {\Gamma \vdash c_1 \equiv c_2 : \kappa_1 \times \kappa_2}
    {\Gamma \vdash \pi_i~c_1 \equiv \pi_i~c_2 : \kappa_i}
\]

\subsection{Motivation}
All of this is useful in the understanding of ML's module system. For example, we wish to
view the type components of a module as a singular type componen\emph{ent}, for which we
must understand a pair of types. Furthermore, functors may be seen as creating type constructors,
for example:

\begin{verbatim}
functor Foo (type t) = struct
  type u = t * t
end
\end{verbatim}
which may be represented as $\lambda (t:\T) \langle t \times t, \ldots \rangle$.

\subsection{Remarks}
\begin{itemize}
  \item We must specify a separate level (kinds) rather than just describing types with types;
    this is described in the
    \href{https://en.wikipedia.org/wiki/Burali-Forti_paradox}{Burali-Forti paradox}.
  \item Counting binders from the inside-out is called de Brujin indices, but counting from
    the outside-in is called de Brujin \emph{levels}, which ``no one uses.''
\end{itemize}

\sectionwithdate{Writing a typechecker for $F_\omega$}{1/18/2018}

In this lecture, we rephrase the declarative syntax in the language of moded
judgments so that we can consider matters of input and output. As before, since
a type may admit at most one kind in $F_\omega$, much of the material introduced
here will stay unnecessary until we have reached the singleton kind calculus.

\subsection{Judgments}

Here is a list of all judgments defined in today's lecture. We use a superscript $+$
to indicate the input and a superscript $-$ to indicate the output.
\begin{align*}
  \Gamma^+ &\vdash e^+ \Rightarrow \tau^- &&\text{type synthesis/inference}\\
  \Gamma^+ &\vdash e^+ \Leftarrow \tau^+ &&\text{type checking/analysis}\\
  \Gamma^+ &\vdash c^+ \Rightarrow \kappa^- &&\text{kind synthesis}\\
  \Gamma^+ &\vdash c^+ \Leftarrow \kappa^+ &&\text{kind checking}\\
  \Gamma^+ &\vdash c+ \Leftrightarrow c'^+ &&\text{algorithmic equivalence}\\
  \Gamma^+ &\vdash q_1^+ \leftrightarrow q_2^+ : \kappa^- &&\text{algorithmic path equivalence}\\
  &c^+ \Downarrow q^- &&\text{weak-head normalization}\\
  &c^+ \leadsto c^- &&\text{weak-head reduction}
\end{align*}

\subsection{Inference Rules}

\begin{judgment}[Type synthesis]
\[ \infer{\Gamma \vdash x \Rightarrow \tau}{\Gamma(x) = \tau} \]
\[
  \infer
    {\Gamma \vdash \lambda (x : \tau).e \Rightarrow \tau \rightarrow \tau'}
    {\Gamma \vdash \tau \Leftarrow \T
    &\Gamma,x : \tau \vdash e \Rightarrow \tau'
    }
  \qquad
  \infer
    {\Gamma \vdash e_1~e_2 \Rightarrow \tau_2}
    {\Gamma \vdash e_1 \Rightarrow \tau
    &\tau \Downarrow \tau_1 \rightarrow \tau_2
    &\Gamma \vdash e_2 \Leftarrow \tau_1
    }
\]
\end{judgment}

Unlike 15-317, we can synthesize the type of a lambda term since we have the type
of the parameter specified in the term.
In the last rule, we don't simply synthesize the type of $e_1$ and confirm that it
is an arrow type.  This is because the type could have redexes. We instead normalize
to $\tau_1 \rightarrow \tau_2$.\footnote{I'm still not clear on why we can rely on WHNF
to correctly ``normalize'' when it seemed to be a theme of lecture that we cannot rely
on this.}

\begin{judgment}[Type checking]
\[
  \infer
    {\Gamma \vdash e \Leftarrow \tau}
    {\Gamma \vdash e \Rightarrow \tau'
    &\Gamma \vdash \tau \Leftrightarrow \tau' : \T
    }
  \qquad
  \infer
    {\Gamma \vdash \Lambda (\alpha : \kappa).e \Rightarrow \forall (\alpha : \kappa).\tau}
    {\Gamma, \alpha : \kappa \vdash e \Rightarrow \tau}
\]
\[
  \infer
    {\Gamma \vdash e[c] \Rightarrow [c/\alpha]\tau'}
    {\Gamma \vdash e \Rightarrow \tau
    &\tau \Downarrow \forall (\alpha : \kappa).\tau'
    &\Gamma \vdash c \Leftarrow \kappa
    }
\]
\end{judgment}

We do not need to check the validity of kinds---\emph{yet} (ominously). The same footnote
holds for the universal type.

\paragraph{Algorithmic equivalence.} An option that works for $F_\omega$ is to repeatedly
contract redeces until reaching a normalized term. However, this doesn't generalize
to the singleton kind calculus, so we develop more machinery here.

\begin{judgment}[Kind synthesis]
  \[ \infer{\Gamma \vdash \alpha \Rightarrow \kappa}{\Gamma(\alpha) = \kappa} \]
  \[\infer{\Gamma \vdash \lambda (\alpha : \kappa).c \Rightarrow \kappa \rightarrow \kappa'}
         {\Gamma, \alpha : \kappa \vdash c \Rightarrow \kappa'}
   \qquad
   \infer{\Gamma \vdash c_1~c_2 \Rightarrow \kappa'}
         {\Gamma \vdash c_1 \Rightarrow \kappa \rightarrow \kappa'
         &\Gamma \vdash c_2 \Leftarrow \kappa
         }
  \]
  \[
    \infer{\Gamma \vdash \langle c_1, c_2 \rangle \Rightarrow \kappa_1 \times \kappa_2}
      {\Gamma \vdash c_1 \Rightarrow \kappa_1
      &\Gamma \vdash c_2 \Rightarrow \kappa_2
      }
    \qquad
    \infer{\Gamma \vdash \pi_i~c \Rightarrow \kappa_i}
      {\Gamma \vdash c \Rightarrow \kappa'
      &\Gamma \vdash c \Rightarrow \kappa_1 \times \kappa_2
      }
  \]
  \[
    \infer{\Gamma \vdash c_1 \rightarrow c_2 \Rightarrow \T}
      {\Gamma \vdash c_1 \Leftarrow \T
      &\Gamma \vdash c_2 \Leftarrow \T
      }
    \qquad
    \infer{\Gamma \vdash \forall(\alpha : \kappa).\tau \Rightarrow \T}
      {\Gamma, \alpha : \kappa \vdash \tau \Leftarrow \T}
  \]
  \[
    \infer[\text{(check)}]
      {\Gamma \vdash c \Leftarrow \kappa}
      {\Gamma \vdash c \Rightarrow \kappa'
      &\kappa = \kappa'
      }
  \]
\end{judgment}

We write $\emph{(check)}$ in such a way that we can more explicitly see what we'll
generalize when we move to other calculi, particularly the notion of equality.

Now when we write the equivalence rules, think about extensionality. We recurse on the
\emph{kind} using the appropriate projection or function application. What we \emph{don't}
do is normalize both types and check equivalence, because this doesn't generalize to
the singleton kind calculus.

\begin{judgment}[Algorithmic equivalence]\thlabel{algequiv}
\[ \infer
    {\Gamma \vdash c \Leftrightarrow c' : \kappa_1 \rightarrow \kappa_2}
    {\Gamma, \alpha : \kappa_1 \vdash c~\alpha \Leftrightarrow c'~\alpha : \kappa_2}
  \qquad
  \infer
    {\Gamma \vdash c \Leftrightarrow c' : \kappa_1 \times \kappa_2}
    {\Gamma \vdash \pi_1~c \Leftrightarrow \pi_1~c' : \kappa_1
    &\Gamma \vdash \pi_2~c \Leftrightarrow \pi_2~c' : \kappa_2
    }
\]
\[
  \infer
    {\Gamma \vdash c \Leftrightarrow c' : \T}
    {c \Downarrow q
    &c' \Downarrow q'
    &\Gamma \vdash q \leftrightarrow q' : \T
    }
\]
\end{judgment}
We can only normalize once we reach a type kind.

A normal form is such that all redeces have been contracted. However, we
only place in \emph{weak head} normal form, where an arrow constructor or
a universal constructor is at the outermost level. (That is, we don't recursively
normalize.)

\begin{judgment}[Weak-head normalization]
\[
  \infer
    {c \Downarrow c''}
    {c \leadsto c' & c' \Downarrow c''}
  \qquad
  \infer
    {c \Downarrow c}
    {c \not\leadsto}
\]
\end{judgment}
Don't be too concerned about $c \not\leadsto$. In practice, we could implement
this in ML by raising an exception if no reduction step is made.

\begin{judgment}[Weak-head reduction]\thlabel{weak}
\[
  \infer{(\lambda(\alpha : \kappa).c)~c' \leadsto [c'/\alpha]c}{}
  \qquad
  \infer{\pi_i~\langle c_1, c_2 \rangle \leadsto c_i}{}
\]
\[
  \infer{c_1~c_2 \leadsto c_1'~c_2}{c_1 \leadsto c_1'}
  \qquad
  \infer{\pi_i~c \leadsto \pi_i~c'}{c \leadsto c'}
\]
\end{judgment}
We only reduce under an application and a projection. This is so that when we
encounter types such as $((\lambda \alpha. \alpha)~(\lambda \alpha.\alpha))~c_3$,
we can reduce them.

The definitions for path and whnf are curiously familiar.
\begin{bnf}
  \text{path}~p \bnfeq
  \alpha
  \alt p~c
  \alt \pi_1~p
  \alt \pi_2~p\\
  \text{whnf}~q \bnfeq
  p
  \alt c \rightarrow c
  \alt \forall (\alpha : \kappa).c
\end{bnf}
Path is clearly a neutral term. Whnf is almost a neutral term, except
the constituents (body of universal type and right/left of arrow may contain
redeces). We have already gotten to kind $\T$, which guarantees that
this grammar is exhaustive.

Algorithmic structural equivalence should look familiar from constructive logic.
The kind is synthesized as an output. Sometimes it is put back in as an input to
the algorithmic equivalence judgment.
\begin{judgment}[Algorithmic structural equivalence]
\[
  \infer{\Gamma \vdash \alpha \leftrightarrow \alpha : \kappa}{\Gamma(\alpha) = \kappa}
  \qquad
  \infer
    {\Gamma \vdash : p~c \leftrightarrow p'~c' : \kappa_2}
    {\Gamma \vdash p \leftrightarrow p' : \kappa_1 \rightarrow \kappa_2
    &\Gamma \vdash c \Leftrightarrow c' : \kappa_1
    }
\]
\[
  \infer
    {\Gamma \vdash \pi_i~p \leftrightarrow \pi_i~p' : \kappa_i}
    {\Gamma \vdash p \leftrightarrow p' : \kappa_1 \times \kappa_2}
  \qquad
  \infer
    {\Gamma \vdash (c_1 \rightarrow c_2) \leftrightarrow (c_1' \rightarrow c_2') : \T}
    {\Gamma \vdash c_1 \Leftrightarrow c_1' : \T
    &\Gamma \vdash c_2 \Leftrightarrow c_2' : \T
    }
\]
\[
  \infer{\Gamma \vdash \forall(\alpha : \kappa).c \leftrightarrow
    \forall(\alpha:\kappa).c' : \T}
    {\Gamma, \alpha : \kappa \vdash c \Leftrightarrow c' : \T}
\]
\end{judgment}

\subsection{Introduction to de Bruijn indices}
With explicit variables, you must be careful to avoid capture. So let's instead
count the number of binders. Substitution coincides with shifting of the indices.
\begin{align*}
  &\uparrow^j_i &&\text{Add $j$ to all variables, except those bound within $i$ binders}\\
  &\uparrow^j_i(\lambda.e) = \lambda.\uparrow^j_{i+1} e\\
  &\uparrow^j_i k = k &&\text{if $k < i$}\\
  &\uparrow^j_i k = k+j &&\text{otherwise}
\end{align*}

\sectionwithdate{Explicit substitution}{1/23/2018}

In this lecture, we introduce a representation of substitutions as mathematical objects
rather than resorting to meta-mathematical reasoning. We adopt the presentation of de
Bruijn indices that will be used in the first project.

\subsection{Intuition and Examples}

Throughout, we use an \emph{ordered context}. As a substitution is being performed,
variables are bound to this context. There may still be free variables whose index
exceeds the number of items in the context; we will have to decrement the indices
of these.

Given the term (in de Bruijn indices)
\[ (\lambda. \lambda. 2 + 0 + 3) [M/0]~~~~\text{,} \]
it would be quite nice if this reduced to
\[ (\lambda. \lambda. M + 0 + 2)~~~~\text{.} \]

Notice again how we talk about substitution as being part of the \emph{term} and of
being \emph{reduced}---this is the first sign that we're modeling substitution
explicitly.

The other context explored today is \emph{shifting} (or \text{lifting}), which involves
yet more terms to the ordered context. Let's give some examples of familiar terms and
their de Bruijn equivalents.
\begin{align}
  w &\vdash \lambda y. \lambda z. z+w && \Longrightarrow && \lambda.\lambda. 0 + 2\\
  w, x & \vdash \lambda y. \lambda z. z + w && \Longrightarrow && \lambda. \lambda. 0 + 3
\end{align}
In (2), we must shift the index of $w$ by 1 to refer to the correct position in
the ordered context.

\subsection{Modelling substitutions explicitly}
We model our approach after \url{http://www.hpl.hp.com/techreports/Compaq-DEC/SRC-RR-54.pdf},
taking some liberties, like indexing from 0.

First, our grammar of terms:
\begin{bnf}
  M \bnfeq i \alt \lambda.M \alt M~M \alt M[\sigma]
\end{bnf}
$i$ denotes de Bruijn indices, and postfix brackets indicate substitution. Now, our grammar
of substitutions $\sigma$.
\begin{bnf}
  \sigma \bnfeq M \cdot \sigma \alt \uparrow^n
\end{bnf}
$\cdot$ functions as cons\footnote{In lecture, we used a period ``.'', but this notation is misleadingly
suggestive of binding.}. We usefully abbreviate $\uparrow^0$ as $\id$, and $\uparrow^1$ as
$\uparrow$.

Substitution is given meaning by equations:
\begin{align*}
  0[M \cdot \sigma] &= M\\
  (i+1)[M \cdot \sigma] &= i[\sigma]\\
  i[\uparrow^n] &= i+n\\
  (M_1~M_2)[\sigma] &= M_1[\sigma]~M_2[\sigma]\\
  (\lambda.M)[\sigma] &= \lambda.M[0 \cdot (\sigma \circ \uparrow)]
\end{align*}

Unlike the reference, we don't model $\circ$ as primitive but rather define it as a binary operator
over substitutions. Our goal is for $M[\sigma \circ \sigma'] = M[\sigma][\sigma']$.
Here's how you compute it, assuming $\cdot$ to bind tighter than $\circ$.
\begin{align*}
  \uparrow^m &\circ \uparrow^n &&= \uparrow^{m+n}\\
  \id &\circ M \cdot \sigma &&= M \cdot \sigma\\
  \uparrow^{m+1} &\circ M \cdot \sigma &&= \uparrow^n \cdot \sigma\\
  M \cdot \sigma &\circ \sigma' &&= M[\sigma'] \cdot (\sigma \circ \sigma')
\end{align*}
Composition here can be pronounced ``before.'' The last rule is the only interesting one, indicating
that later substitutions may depend on prior ones, since we must perform $M[\sigma']$.

For example, we would like the following to hold:
\begin{align*}
  (0+1) [M \cdot \uparrow \circ \uparrow^4] &= (0+1) [M \cdot \uparrow][\uparrow^4]\\
  &= (M + 1)[\uparrow^4]\\
  &= M[\uparrow^4] + 5
\end{align*}
And by our equations, it in fact does.
\begin{align*}
  (0 + 1) [M \cdot \uparrow \circ \uparrow^4] &= (0 + 1) [M[\uparrow^4] \cdot (\uparrow \circ \uparrow^4)]\\
  &= (0 + 1) [M[\uparrow^4] \cdot \uparrow^5]\\
  &= M[\uparrow^4] + 5
\end{align*}

\subsection{Rule conversions}
With these notions of indices and binding, we convert some rules from
from \thref{weak} and \thref{algequiv}. Importantly, the context becomes ordered with this conversion.

\begin{judgment}[Rule conversions with de Bruijn indices]
\[
  \vcenter{
    \infer{(\lambda (\alpha : \kappa). c)~c' \leadsto [c'/\alpha] c}{}
  }
  \qquad
  \Longrightarrow_\emph{de Bruijn}
  \qquad
  \vcenter{
    \infer{(\lambda \kappa.c)~c' \leadsto c[ c' \cdot \id]}{}
  }
\]
\[
  \vcenter{
    \infer
      {\Gamma \vdash c \Leftrightarrow c' : \kappa_1 \rightarrow \kappa_2}
      {\Gamma, \alpha : \kappa_1 \vdash c~\alpha \Leftrightarrow c'~\alpha : \kappa_2}
  }
  \qquad
  \Longrightarrow_\emph{de Bruijn}
  \qquad
  \vcenter{
    \infer
      {\Gamma \vdash c \Leftrightarrow c' : \kappa_1 \rightarrow \kappa_2}
      {\Gamma, \kappa_1 \vdash c[\uparrow]~0 \Leftrightarrow c'[\uparrow]~0 : \kappa_2}
  }
\]
\end{judgment}

\subsection{Substitutions we'll use}
Rather than using substitutions in their full generality, we are really concerned with substitutions
of the form $[ 0 \cdot 1 \cdots n-1 \cdot M_1[\uparrow^n] \cdots M_k[\uparrow^n] \cdot \uparrow^{n+\ell}]$.
This is still pretty general:
\begin{enumerate}[1.]
  \item $[M \cdot \id]$ has $n = 0, k = 1, \ell = 0$.
  \item $[ \uparrow^\ell ]$ has $n = 0, k = 0$.
  \item For $\sigma$ of the desired form, the substitution $[0 \cdot (\sigma \circ \uparrow)]$ retains the form:
    $$0 \cdot (( 0 \cdot 1 \cdots n-1 \cdot M_1[\uparrow^n] \cdots M_k[\uparrow^n] \cdot \uparrow^{n+\ell}) \circ \uparrow)$$
    $$\Downarrow$$
    $$0 \cdot 1 \cdots n \cdot M_1[\uparrow^{n+1}] \cdots M_k[\uparrow^{n+1}] \cdot \uparrow^{n+\ell+1}$$
\end{enumerate}



\end{document}
