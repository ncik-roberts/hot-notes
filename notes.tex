%----------------------------------------------------------------------------------------
%	PACKAGES AND OTHER DOCUMENT CONFIGURATIONS
%----------------------------------------------------------------------------------------

\documentclass[a4paper, 11pt]{article}
\usepackage[margin=1in]{geometry}
\usepackage{amsmath,amsfonts,amssymb}
\usepackage{theoremref}
\usepackage{enumerate}
\usepackage{environ}
\usepackage{hyperref}
\usepackage{xcolor}

\usepackage[protrusion=true,expansion=true]{microtype} % Better typography

\usepackage{mathpazo} % Use the Palatino font
\usepackage[T1]{fontenc} % Required for accented characters
\linespread{1.05} % Change line spacing here, Palatino benefits from a slight increase by default

\DeclareMathOperator{\id}{id}

\makeatletter

% Customize the title
\renewcommand{\maketitle}{
  \noindent{\LARGE\@title}
  \vspace{10pt} % Some vertical space between the title and author name

  \noindent{\@author} % Author name
  \\
  Last built: \@date % Date
  \vspace{5pt}% Some vertical space between the author block and abstract
}

%--------------------------
%   417-specific things
%--------------------------
\usepackage{proof}

% Convenient way to specify date for each lecture
\newcommand{\sectionwithdate}[2]{
  \section{#1 \hfill {\small \textnormal{\emph{(#2)}}}}
}

\NewEnviron{bnf}{
  \begin{align*}
    \BODY
  \end{align*}\ignorespacesafterend
}

\newcommand{\alt}{\; \mid \;}
\newcommand{\altline}{\\&\mid \;}
\newcommand{\bnfeq}{~~&::=~~}
\newcommand{\pack}[3]{\mathtt{pack}[#1,#2]~\mathtt{as}~#3}
\newcommand{\halt}{\mathtt{halt}}
\newcommand{\letv}[3]{\mathtt{let}~#1=#2~\mathtt{in}~#3}
\newcommand{\unpack}[4]{\mathtt{unpack}[#1,#2] = #3~\mathtt{in}~#4}

\renewcommand{\int}{\textnormal{\texttt{int}}}
\newcommand{\unit}{\textnormal{\texttt{unit}}}
\newcommand{\lst}{\textnormal{\texttt{list}}}
\newcommand{\T}{\textnormal{\texttt{T}}} % Type kind
\renewcommand{\S}{\textnormal{\texttt{S}}} % Singleton kind
\newcommand{\kind}{\textnormal{kind}} % Singleton kind
\newcommand{\tarbau}{\overline{\tau}}

%-----------------------
%  Judgment
%-----------------------
\newtheorem{judgment}{Judgment}[section]

%----------------------------------------------------------------------------------------
%	TITLE
%----------------------------------------------------------------------------------------

\title{15-417 HOT Compilation (Spring 2018)}

\author{Scribe: Nick Roberts % Scribe
\\Professor: Karl Crary
\\{\textit{Carnegie Mellon University}}} % Institution

\date{\today} % Date

%----------------------------------------------------------------------------------------

\begin{document}
\maketitle % Print the title section

\sectionwithdate{Definition of $F_\omega$}{1/16/2018}

\emph{Note:} In this section we skip the na\"ive formulation given by Prof. Crary
(without kinds) and immediately introduce kinds.

First, we define type constructors. We often refer to these simply as ``constructors.''
By convention, $\tau$ denotes a nullary constructor, but $c$ may be used in general
without fear.
\begin{bnf}
  c, \tau \bnfeq
  \alpha
  \alt c \rightarrow c
  \alt \forall(\alpha : \kappa). c
  \alt \lambda(\alpha : \kappa). c
  \alt c~c
\end{bnf}

$\alpha$ denotes a type variable. We use these in quantification and type abstraction.
It may be instantiated during application. $\kappa$ denotes a kind, which we now
define:
\begin{bnf}
  \kappa \bnfeq
  \mathtt{type}
  \alt \kappa \rightarrow \kappa
\end{bnf}
Henceforth we use $\T$ to denote \texttt{type}.
Where would we be without terms to inhabit types?
\begin{bnf}
  e \bnfeq x
  \alt \lambda(x : \tau). e
  \alt e ~ e
  \alt \Lambda(\alpha : \kappa). e
  \alt e[\tau]
\end{bnf}

Our context, $\Gamma$, may contain judgments pertaining to types and terms.
\begin{bnf}
  \Gamma \bnfeq \varepsilon
  \alt \Gamma, x : \tau
  \alt \Gamma, \alpha : \kappa
\end{bnf}

Sometimes, for the latter judgment, you will see $\alpha :: \kappa$, but this is
not too important. Proceeding from this context, we first define inductively
the judgment that $\Gamma \vdash c : \kappa$

\begin{judgment}[Type kind]\thlabel{kind}
  $\Gamma \vdash c : \kappa$
\[
  \infer{\Gamma \vdash \alpha : \kappa}{\Gamma(\alpha) : \kappa}
  \qquad
  \infer{\Gamma \vdash c_1 \rightarrow c_2 : \T}
        {\Gamma \vdash c_1 : \T
        &\Gamma \vdash c_2 : \T
        }
  \qquad
  \infer{\Gamma \vdash \forall(\alpha : \kappa).c : \T}
        {\Gamma, \alpha : \kappa \vdash c : \T}
\]
\[
  \infer{\Gamma \vdash c_1 ~ c_2 : \kappa'}
        {\Gamma \vdash c_1 : \kappa \rightarrow \kappa'
        &\Gamma \vdash c_2 : \kappa
        }
  \qquad
  \infer{\Gamma \vdash \lambda(\alpha : \kappa).c : \kappa \to \kappa'}
        {\Gamma, \alpha : \kappa \vdash c : \kappa'}
\]
\end{judgment}

Using this judgment, we next define the judgment $\Gamma \vdash e : \tau$.

\begin{judgment}[Term type]\thlabel{term}
$\Gamma \vdash e : \tau$
\[
  \infer{\Gamma \vdash x : \tau}{\Gamma(x) = \tau}
  \qquad
  \infer{\Gamma \vdash \lambda(x : \tau).e : \tau \rightarrow \tau'}
        {\Gamma, x : \tau \vdash e : \tau'
        &\Gamma \vdash \tau : \kappa
        }
  \qquad
  \infer{\Gamma \vdash e_1 ~ e_2 : \tau'}
        {\Gamma \vdash e_1 : \tau \rightarrow \tau'
        &\Gamma \vdash e_2 : \tau
        }
\]
\[
  \infer{\Gamma \vdash \Lambda (\alpha : \kappa).e : \forall (\alpha : \kappa).\tau}
        {\Gamma, \alpha : \kappa \vdash e : \tau}
  \qquad
  \infer{\Gamma \vdash e[\tau] : [\tau / \alpha]\tau'}
        {\Gamma \vdash e : \forall(\alpha : \kappa).\tau'
        &\Gamma \vdash \tau : \kappa
        }
\]
\end{judgment}

But these rules aren't sufficient: if
$f : \forall(\alpha : \T \rightarrow \T). \alpha~\int \rightarrow \unit$,
we would want \mbox{$f[\lambda(\beta : \T).\beta]~12 : \unit$}, but
currently this is not the case. We need a way to show
\mbox{$(\lambda \beta. \beta) \int \equiv int$}. To do so,
we define a new judgment:
\[\Gamma \vdash c \equiv c' : \kappa\]
and add to \thref{term} the inference rule
\[\infer[(equivalence)]{\Gamma \vdash e : \tau'}
  {\Gamma \vdash e : \tau
  &\Gamma \vdash \tau \equiv \tau' : \T
  }
\]

\begin{judgment}[Equivalence]\thlabel{equiv}
$\Gamma \vdash c \equiv c' : \kappa$

Equivalence rules
\[
  \infer{\Gamma \vdash c \equiv c : \kappa}{\Gamma \vdash c : \kappa}
  \qquad
  \infer{\Gamma \vdash c' \equiv c : \kappa}{\Gamma \vdash c \equiv c' : \kappa}
  \qquad
  \infer{\Gamma \vdash c_1 \equiv c_3 : \kappa}
        {\Gamma \vdash c_1 \equiv c_2 : \kappa
        &\Gamma \vdash c_2 \equiv c_3 : \kappa
        }
\]

Compatibility rules
\[
  \infer{\Gamma \vdash (c_1 \rightarrow c_2) \equiv (c_1' \rightarrow c_2') : \T}
        {\Gamma \vdash c_1 \equiv c_1' : \T
        &\Gamma \vdash c_2 \equiv c_2' : \T
        }
  \qquad
  \infer{\Gamma \vdash \forall(\alpha : \kappa).c \equiv \forall(\alpha : \kappa).c' : \T}
        {\Gamma, \alpha : \kappa \vdash c \equiv c' : \T}
\]
\[
  \infer{\Gamma \vdash \lambda(\alpha : \kappa).c \equiv
          \lambda (\alpha : \kappa.c') \equiv \kappa \rightarrow \kappa'}
       {\Gamma, \alpha : \kappa \vdash c \equiv c' : \kappa'}
  \qquad
  \infer{\Gamma \vdash c_1 ~ c_2 \vdash c_1'~c_2' : \kappa'}
        {\Gamma \vdash c_1 \equiv c_1' : \kappa \rightarrow \kappa'
        &\Gamma \vdash c_2 \equiv c_2' : \kappa
        }
\]
\end{judgment}
These rules defined a congruence relation. We haven't yet added the interesting rules.
Note the enforcement that some constructors be types ($\T$), since they certainly
may not be $\lambda$-abstractions. At this point in the semester, the $\kappa$ constraints
may be omitted, since they are uniquely determined by the \textbf{regularity conditions}
(assuming $\vdash \Gamma~\mathsf{ok}$):
\begin{itemize}
  \item If $\Gamma \vdash c_1 \equiv c_2 : \kappa$, then $\Gamma \vdash c_1 : \kappa$
    and $\Gamma \vdash c_2 : \kappa$.
  \item If $\Gamma \vdash e : \tau$, then $\Gamma \vdash \tau : \T$.
\end{itemize}

Now for a more interesting rules to add to \thref{equiv}:
\[
  \infer[(\beta)]
    {\Gamma \vdash (\lambda (\alpha : \kappa). c') c \equiv [c / \alpha]c' : \kappa'}
    {\Gamma \vdash c : \kappa
    & \Gamma, \alpha : \kappa \vdash c' : \kappa'
    }
\]
We can prove:
\[
  \infer
    {(\lambda \beta. \beta)~\int \rightarrow \unit \equiv \int \rightarrow \unit : \T}
    {\infer
       {(\lambda \beta. \beta)~\int \equiv \int : \T}
       {\infer{\beta : \T \vdash \beta : \T}{}
       &\infer{\int : \T}{}
       }
    }
\]

What about $\eta$-expansion? something along the lines of:
\[
  \infer[(\eta)]
    {\Gamma \vdash c \equiv \lambda (\alpha : \kappa). c~\alpha : \kappa \rightarrow \kappa'}
    {\Gamma \vdash c : \kappa \rightarrow \kappa'}
\]

We want something more general, akin to running an experiment on two constructors of
function kind. (Adding to \thref{equiv}.)
\[
  \infer[(extensionality)]
    {\Gamma \vdash c_1 \equiv c_2 : \kappa \rightarrow \kappa'}
    {\Gamma, \alpha : \kappa \vdash c_1~\alpha \equiv c_2~\alpha : \kappa'}
\]

Exercise: prove $\eta$ from this rule.

\subsection{$F_\omega$ plus products}
\begin{bnf}
  \kappa \bnfeq
    \cdots \alt \kappa \times \kappa\\
  c \bnfeq \cdots \alt \langle c, c \rangle
  \alt \pi_1~c
  \alt \pi_2~c
\end{bnf}

Adding to \thref{kind}:
\[
  \infer{\Gamma \vdash \langle c_1, c_2 \rangle : \kappa_1 \times \kappa_2}
    {\Gamma \vdash c_1 : \kappa_1
    &\Gamma \vdash c_2 : \kappa_2
    }
  \qquad
  \infer{\Gamma \vdash \pi_i~c:\kappa_i}
    {\Gamma \vdash c : \kappa_1 \times \kappa_2}
\]

Adding to \thref{equiv}, the compatability rules:
\[
  \infer{\Gamma \vdash \langle c_1, c_2 \rangle \equiv \langle c_1', c_2' \rangle
    : \kappa_1 \times \kappa_2}
    {\Gamma \vdash c_1 \equiv c_1' : \kappa' & \Gamma \vdash c_2 \equiv c_2' : \kappa_2}
  \qquad
 \infer{\Gamma \vdash \pi_i~c_1 \equiv \pi_i~c_2 : \kappa_i}
   {\Gamma \vdash c_1 \equiv c_2 : \kappa_1 \times \kappa_2}
\]

Adding to \thref{equiv}, the ``interesting'' rules:
\[
  \infer[(\beta_\pi)]
    {\Gamma \vdash \pi_1 \langle c_1, c_2 \rangle \equiv c_1 : \kappa_i}
    {\Gamma \vdash c_1 : \kappa_1 & \Gamma \vdash c_2 : \kappa_2}
  \qquad
  \infer[(extensionality_\pi)]
    {\Gamma \vdash c_1 \equiv c_2 : \kappa_1 \times \kappa_2}
    {\Gamma \vdash \pi_i~c_1 \equiv \pi_i~c_2 : \kappa_i}
\]

\subsection{Motivation}
All of this is useful in the understanding of ML's module system. For example, we wish to
view the type components of a module as a singular type componen\emph{ent}, for which we
must understand a pair of types. Furthermore, functors may be seen as creating type constructors,
for example:

\begin{verbatim}
functor Foo (type t) = struct
  type u = t * t
end
\end{verbatim}
which may be represented as $\lambda (t:\T) \langle t \times t, \ldots \rangle$.

\subsection{Remarks}
\begin{itemize}
  \item We must specify a separate level (kinds) rather than just describing types with types;
    this is described in the
    \href{https://en.wikipedia.org/wiki/Burali-Forti_paradox}{Burali-Forti paradox}.
  \item Counting binders from the inside-out is called de Brujin indices, but counting from
    the outside-in is called de Brujin \emph{levels}, which ``no one uses.''
\end{itemize}

\sectionwithdate{Writing a typechecker for $F_\omega$}{1/18/2018}

In this lecture, we rephrase the declarative syntax in the language of moded
judgments so that we can consider matters of input and output. As before, since
a type may admit at most one kind in $F_\omega$, much of the material introduced
here will stay unnecessary until we have reached the singleton kind calculus.

\subsection{Judgments}

Here is a list of all judgments defined in today's lecture. We use a superscript $+$
to indicate the input and a superscript $-$ to indicate the output.
\begin{align*}
  \Gamma^+ &\vdash e^+ \Rightarrow \tau^- &&\text{type synthesis/inference}\\
  \Gamma^+ &\vdash e^+ \Leftarrow \tau^+ &&\text{type checking/analysis}\\
  \Gamma^+ &\vdash c^+ \Rightarrow \kappa^- &&\text{kind synthesis}\\
  \Gamma^+ &\vdash c^+ \Leftarrow \kappa^+ &&\text{kind checking}\\
  \Gamma^+ &\vdash c+ \Leftrightarrow c'^+ &&\text{algorithmic equivalence}\\
  \Gamma^+ &\vdash q_1^+ \leftrightarrow q_2^+ : \kappa^- &&\text{algorithmic path equivalence}\\
  &c^+ \Downarrow q^- &&\text{weak-head normalization}\\
  &c^+ \leadsto c^- &&\text{weak-head reduction}
\end{align*}

\subsection{Inference Rules}

\begin{judgment}[Type synthesis]
\[ \infer{\Gamma \vdash x \Rightarrow \tau}{\Gamma(x) = \tau} \]
\[
  \infer
    {\Gamma \vdash \lambda (x : \tau).e \Rightarrow \tau \rightarrow \tau'}
    {\Gamma \vdash \tau \Leftarrow \T
    &\Gamma,x : \tau \vdash e \Rightarrow \tau'
    }
  \qquad
  \infer
    {\Gamma \vdash e_1~e_2 \Rightarrow \tau_2}
    {\Gamma \vdash e_1 \Rightarrow \tau
    &\tau \Downarrow \tau_1 \rightarrow \tau_2
    &\Gamma \vdash e_2 \Leftarrow \tau_1
    }
\]
\end{judgment}

Unlike 15-317, we can synthesize the type of a lambda term since we have the type
of the parameter specified in the term.
In the last rule, we don't simply synthesize the type of $e_1$ and confirm that it
is an arrow type.  This is because the type could have redexes. We instead normalize
to $\tau_1 \rightarrow \tau_2$.\footnote{I'm still not clear on why we can rely on WHNF
to correctly ``normalize'' when it seemed to be a theme of lecture that we cannot rely
on this.}

\begin{judgment}[Type checking]
\[
  \infer
    {\Gamma \vdash e \Leftarrow \tau}
    {\Gamma \vdash e \Rightarrow \tau'
    &\Gamma \vdash \tau \Leftrightarrow \tau' : \T
    }
  \qquad
  \infer
    {\Gamma \vdash \Lambda (\alpha : \kappa).e \Rightarrow \forall (\alpha : \kappa).\tau}
    {\Gamma, \alpha : \kappa \vdash e \Rightarrow \tau}
\]
\[
  \infer
    {\Gamma \vdash e[c] \Rightarrow [c/\alpha]\tau'}
    {\Gamma \vdash e \Rightarrow \tau
    &\tau \Downarrow \forall (\alpha : \kappa).\tau'
    &\Gamma \vdash c \Leftarrow \kappa
    }
\]
\end{judgment}

We do not need to check the validity of kinds---\emph{yet} (ominously). The same footnote
holds for the universal type.

\paragraph{Algorithmic equivalence.} An option that works for $F_\omega$ is to repeatedly
contract redeces until reaching a normalized term. However, this doesn't generalize
to the singleton kind calculus, so we develop more machinery here.

\begin{judgment}[Kind synthesis]
  \[ \infer{\Gamma \vdash \alpha \Rightarrow \kappa}{\Gamma(\alpha) = \kappa} \]
  \[\infer{\Gamma \vdash \lambda (\alpha : \kappa).c \Rightarrow \kappa \rightarrow \kappa'}
         {\Gamma, \alpha : \kappa \vdash c \Rightarrow \kappa'}
   \qquad
   \infer{\Gamma \vdash c_1~c_2 \Rightarrow \kappa'}
         {\Gamma \vdash c_1 \Rightarrow \kappa \rightarrow \kappa'
         &\Gamma \vdash c_2 \Leftarrow \kappa
         }
  \]
  \[
    \infer{\Gamma \vdash \langle c_1, c_2 \rangle \Rightarrow \kappa_1 \times \kappa_2}
      {\Gamma \vdash c_1 \Rightarrow \kappa_1
      &\Gamma \vdash c_2 \Rightarrow \kappa_2
      }
    \qquad
    \infer{\Gamma \vdash \pi_i~c \Rightarrow \kappa_i}
      {\Gamma \vdash c \Rightarrow \kappa'
      &\Gamma \vdash c \Rightarrow \kappa_1 \times \kappa_2
      }
  \]
  \[
    \infer{\Gamma \vdash c_1 \rightarrow c_2 \Rightarrow \T}
      {\Gamma \vdash c_1 \Leftarrow \T
      &\Gamma \vdash c_2 \Leftarrow \T
      }
    \qquad
    \infer{\Gamma \vdash \forall(\alpha : \kappa).\tau \Rightarrow \T}
      {\Gamma, \alpha : \kappa \vdash \tau \Leftarrow \T}
  \]
  \[
    \infer[\text{(check)}]
      {\Gamma \vdash c \Leftarrow \kappa}
      {\Gamma \vdash c \Rightarrow \kappa'
      &\kappa = \kappa'
      }
  \]
\end{judgment}

We write $\emph{(check)}$ in such a way that we can more explicitly see what we'll
generalize when we move to other calculi, particularly the notion of equality.

Now when we write the equivalence rules, think about extensionality. We recurse on the
\emph{kind} using the appropriate projection or function application. What we \emph{don't}
do is normalize both types and check equivalence, because this doesn't generalize to
the singleton kind calculus.

\begin{judgment}[Algorithmic equivalence]\thlabel{algequiv}
\[ \infer
    {\Gamma \vdash c \Leftrightarrow c' : \kappa_1 \rightarrow \kappa_2}
    {\Gamma, \alpha : \kappa_1 \vdash c~\alpha \Leftrightarrow c'~\alpha : \kappa_2}
  \qquad
  \infer
    {\Gamma \vdash c \Leftrightarrow c' : \kappa_1 \times \kappa_2}
    {\Gamma \vdash \pi_1~c \Leftrightarrow \pi_1~c' : \kappa_1
    &\Gamma \vdash \pi_2~c \Leftrightarrow \pi_2~c' : \kappa_2
    }
\]
\[
  \infer
    {\Gamma \vdash c \Leftrightarrow c' : \T}
    {c \Downarrow q
    &c' \Downarrow q'
    &\Gamma \vdash q \leftrightarrow q' : \T
    }
\]
\end{judgment}
We can only normalize once we reach a type kind.

A normal form is such that all redeces have been contracted. However, we
only place in \emph{weak head} normal form, where an arrow constructor or
a universal constructor is at the outermost level. (That is, we don't recursively
normalize.)

\begin{judgment}[Weak-head normalization]
\[
  \infer
    {c \Downarrow c''}
    {c \leadsto c' & c' \Downarrow c''}
  \qquad
  \infer
    {c \Downarrow c}
    {c \not\leadsto}
\]
\end{judgment}
Don't be too concerned about $c \not\leadsto$. In practice, we could implement
this in ML by raising an exception if no reduction step is made.

\begin{judgment}[Weak-head reduction]\thlabel{weak}
\[
  \infer{(\lambda(\alpha : \kappa).c)~c' \leadsto [c'/\alpha]c}{}
  \qquad
  \infer{\pi_i~\langle c_1, c_2 \rangle \leadsto c_i}{}
\]
\[
  \infer{c_1~c_2 \leadsto c_1'~c_2}{c_1 \leadsto c_1'}
  \qquad
  \infer{\pi_i~c \leadsto \pi_i~c'}{c \leadsto c'}
\]
\end{judgment}
We only reduce under an application and a projection. This is so that when we
encounter types such as $((\lambda \alpha. \alpha)~(\lambda \alpha.\alpha))~c_3$,
we can reduce them.

The definitions for path and whnf are curiously familiar.
\begin{bnf}
  \text{path}~p \bnfeq
  \alpha
  \alt p~c
  \alt \pi_1~p
  \alt \pi_2~p\\
  \text{whnf}~q \bnfeq
  p
  \alt c \rightarrow c
  \alt \forall (\alpha : \kappa).c
\end{bnf}
Path is clearly a neutral term. Whnf is almost a neutral term, except
the constituents (body of universal type and right/left of arrow may contain
redeces). We have already gotten to kind $\T$, which guarantees that
this grammar is exhaustive.

Algorithmic structural equivalence should look familiar from constructive logic.
The kind is synthesized as an output. Sometimes it is put back in as an input to
the algorithmic equivalence judgment.
\begin{judgment}[Algorithmic structural equivalence]
\[
  \infer{\Gamma \vdash \alpha \leftrightarrow \alpha : \kappa}{\Gamma(\alpha) = \kappa}
  \qquad
  \infer
    {\Gamma \vdash : p~c \leftrightarrow p'~c' : \kappa_2}
    {\Gamma \vdash p \leftrightarrow p' : \kappa_1 \rightarrow \kappa_2
    &\Gamma \vdash c \Leftrightarrow c' : \kappa_1
    }
\]
\[
  \infer
    {\Gamma \vdash \pi_i~p \leftrightarrow \pi_i~p' : \kappa_i}
    {\Gamma \vdash p \leftrightarrow p' : \kappa_1 \times \kappa_2}
  \qquad
  \infer
    {\Gamma \vdash (c_1 \rightarrow c_2) \leftrightarrow (c_1' \rightarrow c_2') : \T}
    {\Gamma \vdash c_1 \Leftrightarrow c_1' : \T
    &\Gamma \vdash c_2 \Leftrightarrow c_2' : \T
    }
\]
\[
  \infer{\Gamma \vdash \forall(\alpha : \kappa).c \leftrightarrow
    \forall(\alpha:\kappa).c' : \T}
    {\Gamma, \alpha : \kappa \vdash c \Leftrightarrow c' : \T}
\]
\end{judgment}

\subsection{Introduction to de Bruijn indices}
With explicit variables, you must be careful to avoid capture. So let's instead
count the number of binders. Substitution coincides with shifting of the indices.
\begin{align*}
  &\uparrow^j_i &&\text{Add $j$ to all variables, except those bound within $i$ binders}\\
  &\uparrow^j_i(\lambda.e) = \lambda.\uparrow^j_{i+1} e\\
  &\uparrow^j_i k = k &&\text{if $k < i$}\\
  &\uparrow^j_i k = k+j &&\text{otherwise}
\end{align*}

\sectionwithdate{Explicit substitution}{1/23/2018}

In this lecture, we introduce a representation of substitutions as mathematical objects
rather than resorting to meta-mathematical reasoning. We adopt the presentation of de
Bruijn indices that will be used in the first project.

\subsection{Intuition and Examples}

Throughout, we use an \emph{ordered context}. As a substitution is being performed,
variables are bound to this context. There may still be free variables whose index
exceeds the number of items in the context; we will have to decrement the indices
of these.

Given the term (in de Bruijn indices)
\[ (\lambda. \lambda. 2 + 0 + 3) [M/0]~~~~\text{,} \]
it would be quite nice if this reduced to
\[ (\lambda. \lambda. M + 0 + 2)~~~~\text{.} \]

Notice again how we talk about substitution as being part of the \emph{term} and of
being \emph{reduced}---this is the first sign that we're modeling substitution
explicitly.

The other context explored today is \emph{shifting} (or \text{lifting}), which involves
yet more terms to the ordered context. Let's give some examples of familiar terms and
their de Bruijn equivalents.
\begin{align}
  w &\vdash \lambda y. \lambda z. z+w && \Longrightarrow && \lambda.\lambda. 0 + 2\\
  w, x & \vdash \lambda y. \lambda z. z + w && \Longrightarrow && \lambda. \lambda. 0 + 3
\end{align}
In (2), we must shift the index of $w$ by 1 to refer to the correct position in
the ordered context.

\subsection{Modelling substitutions explicitly}
We model our approach after \url{http://www.hpl.hp.com/techreports/Compaq-DEC/SRC-RR-54.pdf},
taking some liberties, like indexing from 0.

First, our grammar of terms:
\begin{bnf}
  M \bnfeq i \alt \lambda.M \alt M~M \alt M[\sigma]
\end{bnf}
$i$ denotes de Bruijn indices, and postfix brackets indicate substitution. Now, our grammar
of substitutions $\sigma$.
\begin{bnf}
  \sigma \bnfeq M \cdot \sigma \alt \uparrow^n
\end{bnf}
$\cdot$ functions as cons\footnote{In lecture, we used a period ``.'', but this notation is misleadingly
suggestive of binding.}. We usefully abbreviate $\uparrow^0$ as $\id$, and $\uparrow^1$ as
$\uparrow$.

Substitution is given meaning by equations:
\begin{align*}
  0[M \cdot \sigma] &= M\\
  (i+1)[M \cdot \sigma] &= i[\sigma]\\
  i[\uparrow^n] &= i+n\\
  (M_1~M_2)[\sigma] &= M_1[\sigma]~M_2[\sigma]\\
  (\lambda.M)[\sigma] &= \lambda.M[0 \cdot (\sigma \circ \uparrow)]
\end{align*}

Unlike the reference, we don't model $\circ$ as primitive but rather define it as a binary operator
over substitutions. Our goal is for $M[\sigma \circ \sigma'] = M[\sigma][\sigma']$.
Here's how you compute it, assuming $\cdot$ to bind tighter than $\circ$.
\begin{align*}
  \uparrow^m &\circ \uparrow^n &&= \uparrow^{m+n}\\
  \id &\circ M \cdot \sigma &&= M \cdot \sigma\\
  \uparrow^{m+1} &\circ M \cdot \sigma &&= \uparrow^n \cdot \sigma\\
  M \cdot \sigma &\circ \sigma' &&= M[\sigma'] \cdot (\sigma \circ \sigma')
\end{align*}
Composition here can be pronounced ``before.'' The last rule is the only interesting one, indicating
that later substitutions may depend on prior ones, since we must perform $M[\sigma']$.

For example, we would like the following to hold:
\begin{align*}
  (0+1) [M \cdot \uparrow \circ \uparrow^4] &= (0+1) [M \cdot \uparrow][\uparrow^4]\\
  &= (M + 1)[\uparrow^4]\\
  &= M[\uparrow^4] + 5
\end{align*}
And by our equations, it in fact does.
\begin{align*}
  (0 + 1) [M \cdot \uparrow \circ \uparrow^4] &= (0 + 1) [M[\uparrow^4] \cdot (\uparrow \circ \uparrow^4)]\\
  &= (0 + 1) [M[\uparrow^4] \cdot \uparrow^5]\\
  &= M[\uparrow^4] + 5
\end{align*}

\subsection{Rule conversions}
With these notions of indices and binding, we convert some rules from
from \thref{weak} and \thref{algequiv}. Importantly, the context becomes ordered with this conversion.

\begin{judgment}[Rule conversions with de Bruijn indices]
\[
  \vcenter{
    \infer{(\lambda (\alpha : \kappa). c)~c' \leadsto [c'/\alpha] c}{}
  }
  \qquad
  \Longrightarrow_\emph{de Bruijn}
  \qquad
  \vcenter{
    \infer{(\lambda \kappa.c)~c' \leadsto c[ c' \cdot \id]}{}
  }
\]
\[
  \vcenter{
    \infer
      {\Gamma \vdash c \Leftrightarrow c' : \kappa_1 \rightarrow \kappa_2}
      {\Gamma, \alpha : \kappa_1 \vdash c~\alpha \Leftrightarrow c'~\alpha : \kappa_2}
  }
  \qquad
  \Longrightarrow_\emph{de Bruijn}
  \qquad
  \vcenter{
    \infer
      {\Gamma \vdash c \Leftrightarrow c' : \kappa_1 \rightarrow \kappa_2}
      {\Gamma, \kappa_1 \vdash c[\uparrow]~0 \Leftrightarrow c'[\uparrow]~0 : \kappa_2}
  }
\]
\end{judgment}

\subsection{Substitutions we'll use}
Rather than using substitutions in their full generality, we are really concerned with substitutions
of the form $[ 0 \cdot 1 \cdots n-1 \cdot M_1[\uparrow^n] \cdots M_k[\uparrow^n] \cdot \uparrow^{n+\ell}]$.
This is still pretty general:
\begin{enumerate}[1.]
  \item $[M \cdot \id]$ has $n = 0, k = 1, \ell = 0$.
  \item $[ \uparrow^\ell ]$ has $n = 0, k = 0$.
  \item For $\sigma$ of the desired form, the substitution $[0 \cdot (\sigma \circ \uparrow)]$ retains the form:
    $$0 \cdot (( 0 \cdot 1 \cdots n-1 \cdot M_1[\uparrow^n] \cdots M_k[\uparrow^n] \cdot \uparrow^{n+\ell}) \circ \uparrow)$$
    $$\Downarrow$$
    $$0 \cdot 1 \cdots n \cdot M_1[\uparrow^{n+1}] \cdots M_k[\uparrow^{n+1}] \cdot \uparrow^{n+\ell+1}$$
\end{enumerate}


\sectionwithdate{Singleton Kind Calculus}{1/25/2018}

\subsection{Remarks from previous lecture}
\begin{itemize}
  \item Although in the previous section I treated $M[\sigma]$ as a term, Prof\@. Crary
    wanted to present it as an operation on terms.
  \item $M \cdot \uparrow^k \equiv M \cdot k \cdot k+1 \cdot \cdots$
\end{itemize}

\subsection{Toward the Singleton Kind Calculus}
Starting with $F_\omega$ + products, let's add one more kind scheme.
\begin{bnf}
  \kappa \bnfeq \T
  \alt \kappa \to \kappa
  \alt \kappa \times \kappa
  \alt \S(c)
\end{bnf}
From a set-theoretic point of view, $\S(c)$ is the singleton set $\{ c \}$, but we of
course look down on set theory, so don't go too far with this.

For sure, $\int : \S(\int)$. But we also want that $(\lambda \alpha. \alpha)~\int : \S(\int)$.
(The kind of the parameter $\alpha$ was unspecified in class, but it seems that we want
to allow $\alpha : \T$ in addition to the obvious $\alpha : \S(\int)$.)

Let's review the components of a basic ML signature:
\begin{verbatim}
    type t
    type u
    val a : t
\end{verbatim}
The type component of this signature is $\alpha : \T \times \T$, and the value component
is $a : \pi_2~\alpha$. In an explicit, made-up ML syntax, this would look like:
\begin{verbatim}
    typeconstructor t : type
    typeconstructor u : type
    val a : t
\end{verbatim}

This is nice and breezy, but what happens if we need to introduce sharing constraints
between types? The classic example of the non-geopolitical Diamond Import Problem is where
a lexer and parser signature both reference a symbol table module.
A compiler, needing to make use of both a lexer and a parser, must specify that its lexer
and parser don't deviate in symbol table.  
(A remark was made that Haskell solves this using ``sharing by construction (parameterization),''
in contrast to the ML orthodoxy's ``sharing by specification (fibration).''
There are category-theoretical implications to this all, believe it or not. I don't.)

A complication: how to kind $\lambda(\alpha: \T). \int$? Surely it's $\T \to \T$,
but also it's $\T \to \S(\int)$. Fine, but what about $\lambda(\alpha : \T) : \alpha$?
Again, $\T \to \T$ should groove, but we have no way to assign the kind $\T \to \S(?)$.
This formulation doesn't appear to be entirely satisfactory, so let's nix it and throw
dependent kinds into the mix.

\subsection{Dependent kinds}
We instead adopt this grammar of kinds.
\begin{bnf}
  \kappa \bnfeq \T
  \alt \Pi (\alpha : \kappa). \kappa
  \alt \Sigma (\alpha : \kappa). \kappa
  \alt \S(c)
\end{bnf}

Somehow, we want to set this up so:
\begin{align*}
  \lambda(\alpha : \T) . \alpha &: \Pi(\alpha : \T) . T\\
                &: \Pi(\alpha : \T) . \S(\alpha)
\end{align*}
where the second option is clearly the best. So good that no other type has that kind.

Arrows (exponentiation) are iterated products; pairs (products) are iterated sums.
We can still retain the old notation where there is no dependence.
Where we would say $\langle \int, \int\rangle : \T\times\T$ before, we now say
\mbox{$\langle \int, \int \rangle : \Sigma(\alpha : \T). \S(\alpha)$}, and eruditely.

For a signature with kind $\Sigma(\texttt{t} : \S(\int)) . \S(\texttt{t} \times \texttt{t})$,
look no further than
\begin{verbatim}
     type t = int
     type u = t * t
\end{verbatim}

But we impugn kinds with dependence.
It's possible for a kind to be bad, so we resort to the usual
interrogation tactics.
\begin{judgment}[Innocence of kinds]
  $\Gamma \vdash \kappa : \kind$
  \[
    \infer{\Gamma \vdash \T : \kind}{}
    \qquad
    \infer{\Gamma \vdash \S(\tau) : \kind}
      {\Gamma \vdash \tau : \T}
  \]
  \[
    \infer{\Gamma \vdash \Pi(\alpha:\kappa_1).\kappa_2 : \kind}
      {\Gamma \vdash \kappa_1 : \kind
      &\Gamma, \alpha : \kappa_1 \vdash \kappa_2 : \kind
      }
    \qquad
    \infer{\Gamma \vdash \Sigma(\alpha:\kappa_1).\kappa_2 : \kind}
      {\Gamma \vdash \kappa_1 : \kind
      &\Gamma, \alpha : \kappa_1 \vdash \kappa_2 : \kind
      }
  \]
\end{judgment}

\subsection{Equivalence in kind}
Before, in the judgment $\Gamma \vdash c_1 \equiv c_2 : \kappa$, neither the context
$\Gamma$ nor the kind $\kappa$ was needed. But, just as Mercury
casts capricious shadows on the surface of Mars, so too introduces the fallen
kind new pathways to deceit. Dependent, yes---dependent on sin.
\[\lambda(\alpha : \T).\alpha \stackrel{?}{\equiv} \lambda(\alpha:\T).\int\]

At $\T \to \T$, no. At $\S(\int) \to \T$, yeah. Subkinding helps:
$\S(\int) \to \T \ge \T \to \T$. The kind $\kappa$ is necessary to check
equivalence. The context is clearly necessary for dependent
kinds, and I don't feel like reproducing the example from lecture.

Here, our declarative judgments:
\begin{align*}
  \Gamma &\vdash \kappa : \kind &&\text{kind checking (done!)}\\
  \Gamma &\vdash \kappa_1 \equiv \kappa_2 : \kind &&\text{kind equivalence}\\
  \Gamma &\vdash \kappa_1 \le \kappa_2 &&\text{subkinding}\\
  \Gamma &\vdash c : \kappa &&\text{kind checking for types}\\
  \Gamma &\vdash c_1 \equiv c_2 : \kappa &&\text{type equivalence at a kind}\\
  \hline
  \Gamma &\vdash e : c &&\text{terms}\\
  &\vdash \Gamma~\mathsf{ok} &&\text{context ok-ness}
\end{align*}

Above the line is where the action is. Below, unchanged.
Let's briefly state regularity conditions. If $\vdash \Gamma~\mathsf{ok}$, then:
\begin{enumerate}[(i)]
  \item if $\Gamma \vdash \kappa \equiv \kappa' : \kind$, then $\Gamma \vdash \kappa : \kind$
    and $\Gamma \vdash \kappa : \kind$.
  \item if $\Gamma \vdash \kappa \le \kappa'$, then $\Gamma \vdash \kappa : \kind$
    and $\Gamma \vdash \kappa : \kind$.
  \item if $\Gamma \vdash c : \kappa$, then $\Gamma \vdash \kappa : \kind$.
  \item if $\Gamma \vdash c \equiv c' : \kappa$, then $\Gamma \vdash c : \kappa$
    and $\Gamma \vdash c' : \kappa$.
  \item if $\Gamma \vdash e : \tau$, then $\Gamma \vdash \tau : \T$.
\end{enumerate}

We start with defining kind checking for types.
\begin{judgment}[Kind checking for types]
  $\Gamma \vdash c : \kappa$
  \[
    \infer{\Gamma \vdash \alpha : \kappa}{\Gamma(\alpha) = \kappa}
  \]
  \[
    \infer{\Gamma \vdash \lambda(\alpha : \kappa).c : \Pi(\alpha : \kappa).\kappa'}
      {\Gamma \vdash \kappa : \kind
      &\Gamma, \alpha : \kappa \vdash c : \kappa'
      }
    \qquad
    \infer{\Gamma \vdash c_1~c_2 : [c_2 / \alpha]\kappa'}
      {\Gamma \vdash c_1 : \Pi(\alpha:\kappa).\kappa'
      &\Gamma \vdash c_2 : \kappa
      }
  \]
  \[
    \infer{\Gamma \vdash \langle c_1, c_2 \rangle : \Sigma(\alpha : \kappa_1).\kappa_2}
      {\Gamma, \alpha : \kappa_1 \vdash \kappa_2 : \kind
      &\Gamma \vdash c_1 : \kappa_1
      &\Gamma \vdash c_2 : [c_1/\alpha]\kappa_2
      }
  \]
\end{judgment}
In the elimination rule for dependent products, $\kappa$ and $\kappa'$ live under a different
number of binders. Watch out when implementing de Bruijn indices.

That the dependent sum is dual to the dependent product is to be expected, and is
``standard for dependent types.'' This is not entirely satisfactory, so hopefully more
exposure to this notion will make this clear. (Presumably, $\pi_1$ and $\pi_2$ are
still eliminatory for dependent sums, but this went unmentioned.)

\subsection{Remarks}
\begin{itemize}
\item Why can't the declaration \texttt{type t = int} be interpreted as introducing an
  alias for the type $\int$? Prof\@. Crary's explanation was that the signature with this
  binding is a subkind of the signature specifying only \texttt{type t}, and that the
  alias interpretation is not satisfactory in this regard.
\item For more on the Diamond Import Problem, see Pierce's \emph{Advanced Topics in Types and
  Programming Languages}.
\end{itemize}

\sectionwithdate{Close Encounters of the Singleton Kind}{1/30/2018}

As an introduction to this lecture, we extend \thref{sgk:type} with more rules.
\begin{judgment}[Constructor formation (still incomplete)]\thlabel{sgk:incomplete}
  $\Gamma \vdash c : \kappa$
  \[
    \infer{\Gamma \vdash \alpha : \kappa}{\Gamma(\alpha) = \kappa}
  \]
  \[
    \infer{\Gamma \vdash \lambda(\alpha : \kappa).c : \Pi(\alpha : \kappa).\kappa'}
      {\Gamma \vdash \kappa : \kind
      &\Gamma, \alpha : \kappa \vdash c : \kappa'
      }
    \qquad
    \infer{\Gamma \vdash c_1~c_2 : [c_2 / \alpha]\kappa'}
      {\Gamma \vdash c_1 : \Pi(\alpha:\kappa).\kappa'
      &\Gamma \vdash c_2 : \kappa
      }
  \]
  \[
    \infer{\Gamma \vdash \langle c_1, c_2 \rangle : \Sigma(\alpha : \kappa_1).\kappa_2}
      {\Gamma, \alpha : \kappa_1 \vdash \kappa_2 : \kind
      &\Gamma \vdash c_1 : \kappa_1
      &\Gamma \vdash c_2 : [c_1/\alpha]\kappa_2
      }
    \qquad
  \]
  \[
    \infer{\Gamma \vdash \pi_1~c : \kappa_1}
      {\Gamma \vdash c : \Sigma (\alpha : \kappa_1).\kappa_2}
    \qquad
    \infer{\Gamma \vdash \pi_2~c : [\pi_1~c/\alpha]\kappa_2}
      {\Gamma \vdash c : \Sigma (\alpha : \kappa_1).\kappa_2}
  \]
Same as before:
  \[
    \infer{\Gamma \vdash c_1 \to c_2 : \T}{\Gamma \vdash c_1 : \T & \Gamma \vdash c_2 : \T}
    \qquad
    \infer{\Gamma \vdash \forall(\alpha:\kappa).c : \T}
      {\Gamma \vdash \kappa : \kind
      &\Gamma, \alpha : \kappa \vdash c : \T
      }
    \qquad
    \infer{\Gamma \vdash c : \S(c)}
      {\Gamma \vdash c : \T}
  \]
\end{judgment}
We will add more rules to this as we go.

\subsection{Higher-order singletons}
With no effort, we can say $\int : \S(\int)$. However, we have no notion
of $\lst : \S(\lst)$, even though we can encode this as the judgment
$\lst : \Pi(\alpha : \T). \S(\lst~\alpha)$, where the kind on the RHS
uniquely determines the type.

\paragraph{Higher-order singleton. (Defined notion.)}
\begin{align*}
  \S(c : \T) &= \S(c)\\
  \S(c : \Pi(\alpha : \kappa_1).\kappa_2) &= \Pi(\alpha : \kappa_1).\S(c~\alpha : \kappa_2)\\
  \S(c : \Sigma(\alpha : \kappa_1).\kappa_2) &=
    \S(\pi_1~c : \kappa_1) \times \S(\pi_2~c : [\pi_1~c/\alpha]\kappa_2)\\
  \S(c : \S(c')) &= \S(c)
\end{align*}
It is for ease of proof that we write the third rule non-dependently and the
fourth rule as resolving to $c$ instead of $c'$. The third rule might be nicer in a
de Bruijn setting, since there is no shifting involved with it.

You might like the rules \thref{sgk:incomplete} in their current form, but
these rules aren't sufficient to prove seemingly-trivial theorems. For example,

\[ \infer{\lst : \Pi(\alpha : \T). \S(\lst~\alpha)}
    {\lst : \T \to \T
    &\deduce{\T \to \T \le \Pi(\alpha : \T) . \S(\lst~\alpha)}{\textcolor{red}{\times}}
    }
\]
Therefore, we need something like $\eta$-expansion, which we have by convention
called extensionality rules. In addition to \thref{sgk:incomplete}, we add the rules:
\begin{judgment}[Constructor formation (extensionality)]\thlabel{sgk:extensionality}
  \[
    \infer{\Gamma \vdash c : \Pi(\alpha : \kappa_1).\kappa_2}
      {\Gamma, \alpha : \kappa_1 \vdash c~\alpha : \kappa_2
      &\Gamma \vdash \kappa_1 : \kind
      }
    \qquad
    \infer{\Gamma \vdash c : \Sigma(\alpha : \kappa_1).\kappa_2}
      {\Gamma \vdash \pi_1~c : \kappa_1
      &\Gamma \vdash \pi_2~c : [\pi_1~c/\alpha]\kappa_2
      &\Gamma, \alpha : \kappa_1 \vdash \kappa_2 : \kind
      }
  \]
\end{judgment}

We need these rules to show that:
\begin{align*}
  \text{If} \quad \Gamma \vdash c : \kappa \quad \text{and} \quad &\vdash \Gamma~\mathsf{ok},\\
  \text{then} \quad \Gamma &\vdash \S(c : \kappa) : \kind\\
  \text{and} \quad \Gamma &\vdash c : \S(c : \kappa)
\end{align*}

\begin{judgment}[Kind equivalence (boring)]
  \[
    \infer{\Gamma \vdash \kappa \equiv \kappa : \kind}{\Gamma \vdash \kappa : \kind}
    \qquad
    \infer{\Gamma \vdash \kappa \equiv \kappa' : \kind}
          {\Gamma \vdash \kappa' \equiv \kappa : \kind}
    \qquad
    \infer{\Gamma \vdash \kappa_1 \equiv \kappa_3 : \kind}
          {\Gamma \vdash \kappa_1 \equiv \kappa_2 : \kind
          &\Gamma \vdash \kappa_2 \equiv \kappa_3 : \kind
          }
  \]
  \[
    \infer{\Gamma \vdash \S(c) \equiv \S(c') : \kind}
          {\Gamma \vdash c \equiv c' : \T}
    \qquad
    \infer{
      \deduce
        {\Gamma \vdash \Pi(\alpha : \kappa_1).\kappa_2 \equiv
          \Pi(\alpha : \kappa_1').\kappa_2' : \kind}
        {\Gamma \vdash \Sigma(\alpha : \kappa_1).\kappa_2 \equiv
          \Sigma(\alpha : \kappa_1').\kappa_2' : \kind}}
      {\Gamma \vdash \kappa_1 \equiv \kappa_1' : \kind
      &\Gamma, \alpha : \kappa_1 \vdash \kappa_2 \equiv \kappa_2' : \kind
      }
  \]
\end{judgment}

\begin{judgment}[Subkinding]\mbox{}\\
Preorder:
  \[
    \infer{\Gamma \vdash \kappa \le \kappa'}
      {\Gamma \vdash \kappa \equiv \kappa' : \kind}
    \qquad
    \infer{\Gamma \vdash \kappa_1 \le \kappa_3}
      {\Gamma \vdash \kappa_1 \le \kappa_2
      &\Gamma \vdash \kappa_2 \le \kappa_3
      }
  \]
Other:
  \[
    \infer{\Gamma \vdash \S(c) \le \T}{\Gamma \vdash c : \T}
  \]
  \[
    \infer
      {\Gamma \vdash \Pi(\alpha : \kappa_1).\kappa_2 \le \Pi(\alpha : \kappa_1') . \kappa_2'}
      {\Gamma \vdash \kappa_1' \le \kappa_1
      &\Gamma, \alpha : \kappa_1' \vdash \kappa_2 \le \kappa_2'
      &\Gamma, \alpha : \kappa_1 \vdash \kappa_2 : \kind
      }
  \]
  \[
    \infer
      {\Gamma \vdash \Sigma(\alpha : \kappa_1).\kappa_2 \le \Sigma(\alpha : \kappa_1').\kappa_2'}
      {\Gamma \vdash \kappa_1 \le \kappa_1'
      &\Gamma, \alpha : \kappa_1 \vdash \kappa_2 \le \kappa_2'
      &\Gamma, \alpha : \kappa_1' \vdash \kappa_2' : \kind
      }
  \]
\end{judgment}
The rule for $\Pi$ reminds us that functions are contravariant in the domain and
covariant in the codomain. Notice that we choose the more specific $\alpha : \kappa_1'$
when checking $\kappa_2 \le \kappa_2'$; this is because the wider $\kappa_1$ would not make sense
in $\kappa_2$. But that's why we still must check the validity of $\alpha : \kappa_1$
in $\kappa_2$.

From these subsumption rules, we can derive a few useful results:
\[
  \infer
    {\Gamma \vdash \S(c) \le \S(c')}
    {\Gamma \vdash c \equiv c' : \T}
  \qquad
  \infer
    {\Gamma \vdash c : \S(c')}
    {\Gamma \vdash c \equiv c' : \T}
\]

\begin{judgment}[Constructor equivalence]\mbox{}\\
Equivalence relation:
  \[
    \infer{\Gamma \vdash c \equiv c : \kappa}{\Gamma \vdash c : \kappa}
    \qquad
    \infer{\Gamma \vdash c' \equiv c : \kappa}{\Gamma \vdash c \equiv c'}
    \qquad
    \infer{\Gamma \vdash c_1 \equiv c_3 : \kappa}
      {\Gamma \vdash c_1 \equiv c_2 : \kappa
      &\Gamma \vdash c_2 \equiv c_3 : \kappa
      }
  \]
Compatibility rules:
  \[
    \infer{\Gamma \vdash \lambda(\alpha : \kappa_1).c \equiv \lambda(\alpha : \kappa_1').c'
      : \Pi(\alpha : \kappa_1).\kappa_2}
      {\Gamma \vdash \kappa_1 \equiv \kappa_1' : \kind
      &\Gamma, \alpha : \kappa_1 \vdash c \equiv c' : \kappa_2
      }
    \qquad
    \infer{\Gamma \vdash c_1~c_2 \equiv c_1'~c_2' : [c_2/\alpha]\kappa_2}
      {\Gamma \vdash c_1 \equiv c_1' : \Pi(\alpha : \kappa_1).\kappa_2
      &\Gamma \vdash c_2 \equiv c_2' : \kappa_2
      }
  \]
  \[
    \infer{\Gamma \vdash \langle c_1, c_2 \rangle \equiv \langle c_1', c_2' \rangle
      : \Sigma(\alpha : \kappa_1).\kappa_2}
      {\Gamma \vdash c_1 \equiv c_1' : \kappa_1
      &\Gamma \vdash c_2 \equiv c_2' : [c_1 / \alpha]\kappa_2
      }
    \qquad
    \infer{
      \deduce
         {\Gamma \vdash \pi_2~c \equiv \pi_2 c' : [\pi_1~c/\alpha]\kappa_2}
         {\Gamma \vdash \pi_1~c \equiv \pi_1 c' : \kappa_1}
      }
      {\Gamma \vdash c \equiv c' : \Sigma(\alpha : \kappa_1).\kappa_2}
  \]
  Type constructor rules:
  \[
    \infer[(\beta)]
      {\Gamma \vdash (\lambda (\alpha : \kappa_1). c_2)~c_1 \equiv
        [c_1/\alpha]c_2 : [c_1/\alpha]\kappa_2}
      {\Gamma \vdash c_1 : \kappa_1
      &\Gamma, \alpha : \kappa_1 \vdash c_2 : \kappa_2
      }
  \]
  \[
    \infer{\Gamma \vdash \pi_1~\langle c_1, c_2 \rangle \equiv c_1 : \kappa_1}
      {\Gamma \vdash c_1 : \kappa_1
      &\Gamma \vdash c_2 : \kappa_2
      }
    \qquad
    \infer{\Gamma \vdash \pi_2~\langle c_1, c_2 \rangle \equiv c_2 : \kappa_2}
      {\Gamma \vdash c_1 : \kappa_1
      &\Gamma \vdash c_2 : \kappa_2
      }
  \]
  \[
    \infer{\Gamma \vdash c_1 \to c_2 \equiv c_1' \to c_2' : \T}
      {\Gamma \vdash c_1 \equiv c_1' : \T
      &\Gamma \vdash c_2 \equiv c_2' : \T
      }
    \qquad
    \infer{\Gamma \vdash \forall (\alpha : \kappa).c \equiv \forall(\alpha : \kappa').c' : \T}
      {\Gamma \vdash \kappa \equiv \kappa' : \kind
      &\Gamma, \alpha : \kappa \vdash c \equiv c' : \T
      }
  \]
Extensionality rules:
\[
  \infer{\Gamma \vdash c \equiv c' : \Pi(\alpha : \kappa_1) \kappa_2}
    {\Gamma \vdash \kappa_1 : \kind
    &\Gamma, \alpha : \kappa_1 \vdash c~\alpha \equiv c'~\alpha
    &\deduce{\left[ \Gamma \vdash c \equiv c' : \Pi(\alpha : \kappa_1).\kappa_2' \right]}
      {
        \left[\deduce{\Gamma \vdash c : \Pi(\alpha : \kappa_1).\kappa_2'}
              {\Gamma \vdash c' : \Pi(\alpha : \kappa_1).\kappa_2''}\right]
      }
    }
\]
\[
  \infer{\Gamma \vdash c \equiv c' : \Sigma(\alpha : \kappa_1).\kappa_2}
    {\Gamma \vdash \pi_1~c \equiv \pi_1~c' : \kappa_1
    &\Gamma \vdash \pi_2~c \equiv \pi_2~c' : [\pi_1~c/\alpha]\kappa_2
    &\Gamma, \alpha : \kappa_1 \vdash \kappa_2 : \kind
    }
\]
\end{judgment}
The $\forall$ type constructor rule is the only reason we had to define kind equivalence; for the
other rules where we invoke equivalence, we could have simply used structural equality of kinds.
The bracketed premises in the $\Pi$ extensionality rule are the so-called ``nuisance premises''
needed for the proofs to go through. There are two sets of bracketed premises, so there are
actually two rules differing only in which set of bracketed premises you choose.\\

\noindent\emph{Singletons:}
\[
  \infer{\Gamma \vdash c \equiv c' : \kappa'}
    {\Gamma \vdash c \equiv c' : \kappa
    &\Gamma \vdash \kappa \le \kappa'
    }
  \qquad
  \infer[\star]{\Gamma \vdash c \equiv c' : \T}{\Gamma \vdash c : \S(c')}
  \qquad
  \infer{\Gamma \vdash c \equiv c' : \S(c)}{\Gamma \vdash c \equiv c' : \T}
\]

Phew!
For term formation, only one rule changes: checking kind ok-ness for
type abstraction.

(We went through some examples in lecture, but nothing that doesn't follow from the rules
given above. Maybe I'll add them later.)

%TODO: Add examples

\sectionwithdate{Algorithmic Rules for the SKC}{2/1/2018}

\subsection{Judgments}
These should look familiar. We update them to account for singleton kinds.

\begin{align*}
  \Gamma^+ &\vdash e^+ \Rightarrow \tau^- &&\text{type synthesis/inference}\\
  \Gamma^+ &\vdash e^+ \Leftarrow \tau^+ &&\text{type checking/analysis}\\
  \Gamma^+ &\vdash c^+ \Rightarrow \kappa^- &&\text{kind synthesis}\\
  \Gamma^+ &\vdash c^+ \Leftarrow \kappa^+ &&\text{checking kind}\\
  \Gamma^+ &\vdash \kappa^+ \Leftarrow \kind &&\text{kind checking (a kind)}\\
  \Gamma^+ &\vdash \kappa^+ \trianglelefteq \kappa'^+ &&\text{subkinding}\\
  \Gamma^+ &\vdash c^+ \Leftrightarrow c'^+ &&\text{algorithmic equivalence}\\
  \Gamma^+ &\vdash q_1^+ \leftrightarrow q_2^+ : \kappa^- &&\text{algorithmic path equivalence}\\
  \Gamma^+ &\vdash c^+ \Downarrow q^- &&\text{weak-head normalization}\\
  \Gamma^+ &\vdash c^+ \leadsto c^- &&\text{weak-head reduction}\\
  \Gamma^+ &\vdash p^+ \uparrow \kappa^- &&\text{natural kind}
\end{align*}

The main thing to note is that we now need a context for weak-head reduction, since
it tells us which variables are bound to which singletons.

\subsection{Terms}
For type checking, we import \thref{2:ty-check} directly with no change.

For type inference, we make superficial changes to \thref{2:ty-synth}. Here's a representative
updated rule, with a context for weak-head normalization:

\[ \infer{\Gamma \vdash e_1~e_2 \Rightarrow \tau'}
        {\Gamma \vdash e_1 \Rightarrow \tau_1
        &\Gamma \vdash \tau_1 \Downarrow \tau \rightarrow \tau'
        &\Gamma \vdash e_2 \Leftarrow \tau
        }
\]

We also check that the kind is indeed virtuous in the type abstraction term.
\[
  \infer{\Gamma \vdash \Lambda(\alpha : \kappa).e \Rightarrow \forall(\alpha : \kappa).\tau}
    {\Gamma \vdash \kappa \Leftarrow \kind
    &\Gamma, \alpha : \kappa \vdash e \Rightarrow \tau
   }
\]

\subsection{Kinds and types}
\begin{judgment}[Checking kind]
  $\Gamma \vdash \kappa \Leftarrow \kind$
  \[
    \infer{\Gamma \vdash \T \Leftarrow \kind}{}
    \qquad
    \infer{\Gamma \vdash \S(c) \Leftarrow \kind}
          {\Gamma \vdash c \Leftarrow \T}
  \]
  \[
    \infer{
      \deduce
        {\Gamma \vdash \Pi(\alpha : \kappa_1).\kappa_2 \Leftarrow \kind}
        {\Gamma \vdash \Sigma(\alpha : \kappa_1).\kappa_2 \Leftarrow \kind}
      }
      {\Gamma \vdash \kappa_1 \Leftarrow \kind
      &\Gamma, \alpha : \kappa_1 \Leftarrow\kind
      }
  \]
\end{judgment}

\begin{judgment}[Kind checking]
  $\Gamma \vdash c \Leftarrow \kappa$
  \[
    \infer
      {\Gamma \vdash c \Leftarrow \kappa}
      {\Gamma \vdash c \Rightarrow \kappa'
      &\Gamma \vdash \kappa' \trianglelefteq \kappa
      }
  \]
\end{judgment}

\begin{judgment}[Kind synthesis]
  $\Gamma \vdash c \Rightarrow \kappa$
  \[
    \infer[\star]
      {\Gamma \vdash \alpha \Rightarrow \S(\alpha : \kappa)}
      {\Gamma(\alpha) = \kappa}
  \]
  \[
    \infer{\Gamma \vdash \lambda(\alpha : \kappa).c \Rightarrow \Pi(\alpha : \kappa). \kappa'}
      {\Gamma, \alpha : \kappa \vdash c \Rightarrow \kappa'
      &\Gamma \vdash \kappa \Leftarrow \kind
      }
    \qquad
    \infer{\Gamma \vdash c_1~c_2 \Rightarrow [c_2 / \alpha] \kappa'}
      {\Gamma \vdash c_1 \Rightarrow \Pi(\alpha : \kappa). \kappa'
      &\Gamma \vdash c_2 \Leftarrow \kappa
      }
  \]
  \[
    \infer[\star]{\Gamma \vdash \langle c_1, c_2 \rangle : \kappa_1 \times \kappa_2}
      {\Gamma \vdash c_1 \Rightarrow \kappa_1
      &\Gamma \vdash c_2 \Rightarrow \kappa_2
      }
    \qquad
    \infer{\Gamma \vdash \pi_1~c \Rightarrow \kappa_1}
      {\Gamma \vdash c \Rightarrow \Sigma(\alpha : \kappa_1).\kappa_2}
    \qquad
    \infer{\Gamma \vdash \pi_2~c \Rightarrow [\pi_1~c/\alpha]\kappa_2}
      {\Gamma \vdash c \Rightarrow \Sigma(\alpha : \kappa_1).\kappa_2}
  \]
  \[
    \infer{\Gamma \vdash \tau_1 \rightarrow \tau_2 \Rightarrow \S(\tau_1 \rightarrow \tau_2)}
      {\Gamma \vdash \tau_1 \Leftarrow \T
      &\Gamma \vdash \tau_2 \Leftarrow \T
      }
    \qquad
    \infer
      {\Gamma \vdash \forall (\alpha: \kappa).\tau \Rightarrow \S(\forall(\alpha : \kappa).\tau)}
      {\Gamma \vdash \kappa \Leftarrow \kind
      &\Gamma, \alpha : \kappa \vdash \tau \Leftarrow \T
      }
  \]
\end{judgment}

The singleton rule is where the action is. We don't just report the kind of $\alpha$ as
$\T$; we want to report the \emph{principal} kind of $\alpha$.

In the application rule, notice that we don't weak-head normalize the kind inferred for
$c_1$. This is because there's only one way for a dependent product kind to look. (The version
of this statement for $\T$ doesn't hold, of course, because of singleton kinds.)

For the introductory form for product, you might be alarmed that we're not using a dependent sum.
Don't be---there's always a way to write the kind of a pair without the dependent sum. However,
when dealing with, say, a variable given a dependent sum kind, we still need the machinery to
deal with that. But be careful to correctly shift de Bruijn indices with this rule.

\begin{judgment}[Subkinding]
  $\Gamma \vdash \kappa_1 \trianglelefteq \kappa_2$
  \[
    \infer{\Gamma \vdash \T \trianglelefteq \T}{}
    \qquad
    \infer[\star]{\Gamma \vdash \S(c) \trianglelefteq \T}{}
  \]
  \[
    \infer{\Gamma \vdash \S(c) \trianglelefteq \S(c')}{\Gamma \vdash c \Leftrightarrow c' : \T}
  \]
  \[
    \infer{\Gamma \vdash \Sigma(\alpha : \kappa_1). \kappa_2 \trianglelefteq
      \Sigma(\alpha : \kappa_1').\kappa_2'}
      {\Gamma \vdash \kappa_1 \trianglelefteq \kappa_1'
      &\Gamma, \alpha : \kappa_1 \vdash \kappa_2 \trianglelefteq \kappa_2'
      }
    \qquad
    \infer{\Gamma \vdash \Pi(\alpha : \kappa_1). \kappa_2 \trianglelefteq \Pi(\alpha: \kappa_1'). \kappa_2'}
    {\Gamma \vdash \kappa_1' \trianglelefteq \kappa_1
    &\Gamma, \alpha : \kappa_1' \vdash \kappa_2 \trianglelefteq \kappa_2'
    }
  \]
\end{judgment}

These algorithmic rules don't contain provisions for kind-goodness checks in the same way
that the declarative rules do. We only perform such checks when evaluating a kind provided by
the programmer, such as in the $\Lambda$ term. We omit these checks because it's sound to do so.
I omit the list of soundness rules because they are only interesting if you're proving properties
about these rules, which I am not.

%TODO: Stop being a baby and transcribe the soundness rules

\begin{judgment}[Algorithmic equivalence] $\Gamma \vdash c \Leftrightarrow c' : \kappa$
  \[
    \infer[\star]{\Gamma \vdash c \Leftrightarrow c' : \S(c'')}{}
  \]
  \[
    \infer[\star]
      {\Gamma \vdash c \Leftrightarrow c' : \Pi(\alpha : \kappa_1). \kappa_2}
      {\Gamma, \alpha : \kappa_1 \vdash c~\alpha \Leftrightarrow c"~\alpha : \kappa_2}
    \qquad
    \infer
      {\Gamma \vdash c \Leftrightarrow c' : \Sigma(\alpha : \kappa_1). \kappa_2}
      {\Gamma \vdash \pi_1~c \Leftrightarrow \pi_1 c' : \kappa_1
      &\Gamma \vdash \pi_2~c \Leftrightarrow \pi_2c' : [\pi~c/\alpha]\kappa_2
      }
  \]
  \[
    \infer{\Gamma \vdash c_1 \Leftrightarrow c_2 : \T}
      {\Gamma \vdash c_1 \Downarrow c_1'
      &\Gamma\vdash c_2 \Downarrow c_2'
      &\Gamma \vdash c_1' \leftrightarrow c_2' : \T
      }
  \]
\end{judgment}

The singleton rule follows from regularity (i.e. soundness). The other starred rule is starred
so you know to watch out for shifting indices.

The kind has gotten smaller throughout this judgment. However, when dealing with structural
equivalence, the kind actually gets \emph{larger}, and that's why it's so dang hard to prove
termination for the algorithm described by these rules.

Weak-head normalization is the same as \thref{2:whn}, but now threading a context $\Gamma$
through the premises and conclusions.

\begin{judgment}[Weak-head reduction]
  $\Gamma \vdash c \leadsto c'$
  \[
    \infer{\Gamma \vdash (\lambda(\alpha : \kappa).c)~c' \leadsto [c'/\alpha]c}{}
    \qquad
    \infer{\Gamma \vdash c_1~c_2 \leadsto c_1'~c_2}{\Gamma \vdash c_1 \leadsto c_1'}
  \]
  \[
    \infer{\Gamma \vdash \pi_i~\langle c_1, c_2 \rangle \leadsto c_i}{}
    \qquad
    \infer{\Gamma \vdash \pi_i~c \leadsto \pi_i~c'}{\Gamma\vdash c \leadsto c'}
  \]
  \[
    \infer[\star]{\Gamma \vdash c \leadsto c'}{\Gamma \vdash c \uparrow \S(c')}
  \]
\end{judgment}

We want the starred rule to only apply to paths, since otherwise it's wasteful. You can
imagine this rule as looking up the path in the context, but as stupidly as possible. Don't
be clever.

\begin{judgment}[Natural kind. Don't be clever.]
  $\Gamma \vdash p \uparrow \kappa$
  \[
    \infer{\Gamma \vdash \alpha \uparrow \kappa}{\Gamma(\alpha) = \kappa}
    \qquad
    \infer{\Gamma \vdash p~c \uparrow [c/\alpha]\kappa_2}
      {\Gamma \vdash p \uparrow \Pi(\alpha : \kappa_1).\kappa_2}
    \qquad
    \infer{\Gamma \vdash p \uparrow \Sigma(\alpha : \kappa_1).\kappa_2}
      {\deduce
        {\Gamma \vdash \pi_1~p \uparrow \kappa_1}
        {\Gamma \vdash \pi_2~p \uparrow [\pi_1~p/\alpha]\kappa_2}
      }
  \]
\end{judgment}

Remember that whnf is either a lambda, a $\forall$-type, or a path. We don't want $\uparrow$ rules
for lambda or $\forall$, since $\Gamma \vdash \lambda \cdots \uparrow \T$, which is not even
what we're looking for with the natural kind rules. There's also no need to look up
a natural kind for $c_1$ when performing the step $c_1~c_2 \leadsto c_1'~c_2$---this
constructor can never be a singleton, so why bother looking?

There is also a notion of completeness for these algorithmic rules with respect to the
declarative rules from last lecture.

\emph{Note:} Structural equivalence was not covered. We resume with that next lecture.

\sectionwithdate{A Taste of CPL in CPS}{2/6/2018}

First, let's tie up a loose end: structural equivalence of paths.

\begin{judgment}[Structural equivalence]
  $\Gamma \vdash  p \leftrightarrow p' : \kappa$
  \[
    \infer{\Gamma \vdash \alpha \leftrightarrow \alpha : \kappa}
      {\Gamma(\alpha) = \kappa}
  \]
  \[
    \infer[\star]{\Gamma \vdash p~c \leftrightarrow p'~c' : [c/\alpha]\kappa_2}
      {\Gamma \vdash p \leftrightarrow p' : \Pi(\alpha : \kappa_1).\kappa_2
      &\Gamma \vdash c \Leftrightarrow c' : \kappa_1
      }
  \]
  \[
    \infer{\Gamma \vdash \pi_1 p \leftrightarrow \pi_1 p' : \kappa_1}
      {\Gamma \vdash p \leftrightarrow p' : \Sigma(\alpha : \kappa_1).\kappa_2}
    \qquad
    \infer{\Gamma \vdash \pi_2p \leftrightarrow \pi_2p' : [\pi p/\alpha]\kappa_2}
      {\Gamma \vdash p \leftrightarrow p' : \Sigma(\alpha : \kappa_1).\kappa_2}
  \]
  \[ \infer{\Gamma \vdash (c_1 \rightarrow c_2) \leftrightarrow (c_1' \rightarrow c_2') : \T}
      {\Gamma \vdash c_1 \Leftrightarrow c_2' : \T
      &\Gamma \vdash c_2 \Leftrightarrow c_2' : \T
      }
    \qquad
    \infer{\Gamma \vdash \forall(\alpha : \kappa).c \leftrightarrow
            \forall(\alpha : \kappa').c' : \T}
      {\Gamma \vdash \kappa \Leftrightarrow \kappa' : \kind
      &\Gamma, \alpha : \kappa \vdash c \Leftrightarrow c' : \T
      }
  \]
\end{judgment}
In the starred rule, we could have either substituted $c$ or $c'$ for $\alpha$; either
way, the proof is made annoying. We introduced one more judgment along the way:
\begin{judgment}[Kind equivalence]
  $\Gamma \vdash \kappa \Leftrightarrow \kappa' : \kind$
  \[
    \infer{\Gamma \vdash \T \Leftrightarrow \T : \kind}{}
    \qquad
    \infer{\Gamma \vdash \S(c) \Leftrightarrow \S(c') : \kind}
      {\Gamma \vdash c \Leftrightarrow c' : \T}
  \]
  \[
    \infer
      {\deduce
        {\Gamma \vdash \Sigma(\alpha : \kappa_1).\kappa_2 \Leftrightarrow
          \Sigma(\alpha : \kappa_1').\kappa_2' : \kind}
        {\Gamma \vdash \Pi(\alpha : \kappa_1).\kappa_2 \Leftrightarrow
          \Pi(\alpha : \kappa_1').\kappa_2' : \kind}
      }
      {\Gamma \vdash \kappa_1 \Leftrightarrow \kappa_1' : \kind
      &\Gamma, \alpha : \kappa_1 \vdash \kappa_2 \Leftrightarrow \kappa_2' : \kind
      }
  \]
\end{judgment}
Done!

\subsection{CPS conversion}
IL-Direct is roughly the core ML language. In this phase, we will translate IL-Direct
to IL-CPS using, well, CPS conversion.

\paragraph{IL-Direct}
\begin{bnf}
  \tau \bnfeq
  \alpha
  \alt \tau \rightarrow \tau
  \alt \forall(\alpha : \kappa). \tau
  \alt \tau \times \tau
  \alt \exists(\alpha : \kappa).\tau\\
  e \bnfeq x \alt \lambda(x:\tau).e \alt \cdots
\end{bnf}

\paragraph{IL-CPS}
\begin{bnf}
  \kappa \bnfeq \textit{(same as SKC)}\\
  c \bnfeq \textit{(same as SKC)}
  \alt c \times c
  \alt \exists(\alpha : \kappa).c
  \alt \neg c
  \alt \unit\\
  v \bnfeq x \alt \langle v, v \rangle \alt
    \pack{c}{v}{\exists (\alpha : \kappa).\tau}
    \alt \lambda(x : \tau).e \alt \ast\\
  e \bnfeq
    \letv{x}{v}{e}
    \alt \letv{x}{\pi_i v}{e}
    \alt \unpack{\alpha}{x}{v}{e}
    \alt v~v
    \alt \halt
\end{bnf}

Two typing judgments are our main concern at the term level.

\begin{judgment}[Value typing]
  $\Gamma \vdash v : \tau$
  \[
    \infer{\Gamma \vdash x : \tau}{\Gamma(x) = \tau}
    \qquad
    \infer{\Gamma \vdash \langle v_1, v_2 \rangle : \tau_1 \times \tau_2}
      {\Gamma \vdash v_1 : \tau_1
      &\Gamma \vdash v_2 : \tau_2
      }
    \qquad
    \infer{\Gamma \vdash \ast : \unit}{}
  \]
  \[
    \infer{\Gamma \vdash \pack{c}{v}{\exists(\alpha : \kappa).\tau}
      : \exists(\alpha : \kappa).\tau}
      {\Gamma \vdash c : \kappa
      &\Gamma \vdash v : [c/\alpha]\tau
      &\Gamma, \alpha : \kappa \vdash \tau : \T
      }
      \qquad
    \infer{\Gamma \vdash \lambda(x : \tau).e : \neg \tau}
      {\Gamma, x : \tau \vdash e : \mathbf 0
      &\Gamma \vdash \tau : \T
      }
  \]
\end{judgment}

\begin{judgment}[Expression typing]
  $\Gamma \vdash e : \mathbf 0$
  \[
    \infer{\Gamma \vdash \letv x v e : \mathbf{0}}
      {\Gamma \vdash v : \tau
      &\Gamma, x : \tau \vdash e : \mathbf 0
      }
    \qquad
    \infer
      {\Gamma \vdash \letv x {\pi_i v} e: \mathbf 0}
      {\Gamma \vdash v : \tau_1 \times \tau_2
      &\Gamma \vdash x : \tau_I \vdash e : \mathbf 0
      }
  \]
  \[
    \infer
      {\Gamma \vdash \unpack \alpha x v e : \mathbf 0}
      {\Gamma \vdash v : \exists (\alpha : \kappa).\tau
      &\Gamma, \alpha : \kappa, x : \tau \vdash e : \mathbf{0}
      }
  \]
  \[
    \infer
      {\Gamma \vdash v_1~v_2 : \mathbf{0}}
      {\Gamma \vdash v_1 : \neg \tau
      & \Gamma \vdash v_2 : \tau
      }
    \qquad
    \infer{\Gamma \vdash \halt : \mathbf{0}}{}
  \]
\end{judgment}

John Reynolds remarked in his paper \emph{Definitional interpreters for higher-order
programming languages} that CPS conversion resolves any ambiguities in control flow. For example,
it's immediately apparent whether the semantics is call-by-value or call-by-name. We can remark
on additional aspects of CPS conversion. It:
\begin{enumerate}[1.]
  \item names all intermediate computations,
  \item makes control flow explicit, and
  \item reifies continuations.
\end{enumerate}

Many compilers do 1 and 2 in what's known as A-normal form, not to be confused with anormal
form or a normal form. A-normal form is, formally, ``A it until you can't A it anymore.''
A more descriptive name might be monadic form, or two-thirds CPS.

\subsection{Typing derivation--directed translation}
We don't want to do syntax-directed translation, since this would involve putting type
information everywhere in IL-Direct. Here's an example of a type-directed translation
to warm up to IL-CPS.
We define $\overline{\cdot} : \text{IL-$X$ types} \to \text{IL-$Y$ types}$ so that
\begin{align*}
  \text{if}\quad &\tau \quad \text{is an IL-$X$ type,}\\
  \text{then}\quad &\taubar \quad \text{is an IL-$Y$ type.}
\end{align*}

Under this definition, we want to define a judgment
\[ \Gamma \vdash e : \tau \leadsto \ebar \]
such that, if $\Gamma \vdash e : \tau$, then $\Gamma\vdash : \tau \leadsto \ebar$ such that
(roughly) $\Gammabar \vdash \ebar : \taubar$. Just to reiterate: the translation
of $\tau$ to $\taubar$ is syntax-directed; $\ebar$ is just a meta-variable that we will give
meaning to with the translation judgment.

\subsection{Stupidification}
In this section, we define a stupid translation that does nothing meaningful.
\begin{align*}
  \overline{\alpha} &= \alpha &\overline{\T} &= \T\\
  \overline{\tau_1 \to \tau_2} &= \taubar_1 \to \taubar_2 &\overline{\S(c)} &= \S(c)\\
  \overline{\tau_1 \times \tau_2} &= \taubar_1 \times (\taubar_1 \to \taubar_2) &\overline{\Pi(\alpha : \kappa_1).\kappa_2} &= \Pi(\alpha : \kappabar_1).\kappabar_2\\
  &&\overline{\Sigma(\alpha : \kappa_1).\kappa_2} &= \Sigma(\alpha : \kappabar_1).\kappabar_2
\end{align*}
\begin{align*}
  \overline{\varepsilon} &= \varepsilon\\
  \overline{\Gamma, \alpha : \kappa} &= \Gammabar, \alpha : \kappabar\\
  \overline{\Gamma, x : \tau} &= \Gammabar, x : \taubar
\end{align*}

I ellide the rules defining the translation judgment that just recursively translate the
constituent parts. Here are some of the more interesting rules:
\[
  \infer
    {\Gamma \vdash \langle e_1, e_2 \rangle : \tau_1 \times \tau_2
      \leadsto \langle \ebar_1, \lambda(\_ : \taubar_1) : \ebar_2}
    {\Gamma \vdash e_1 : \tau_1 \leadsto \ebar_1
    &\Gamma \vdash e_2 : \tau_2 \leadsto \ebar_2}
  \qquad
  \infer
    {\Gamma \vdash \pi_2e : \tau_2 \leadsto \letv{x}{\ebar}{\pi_2 x(\pi_1 x)}}
    {\Gamma \vdash e : \tau_1 \times \tau_2 \leadsto \ebar}
\]
Now you see why it's important that the translation be type-directed instead of syntax-directed:
otherwise, we wouldn't have known that $\taubar_1$ in the pair introduction rule.

What are some properties we'd like?
\paragraph{(Effectiveness.)}
\begin{enumerate}[1.]
  \item If $c$ is a constructor, then $\overline{c}$ is defined.
    (Likewise for $\kappa$ and $\Gamma$.)
  \item $\Gamma \vdash e : \tau$ is derivable iff $\Gamma \vdash e : \tau \leadsto \ebar$
    for some $\ebar$.
\end{enumerate}
\paragraph{(Static correctness.)}
\begin{enumerate}[1.]
  \item If $\vdash \Gamma~\mathsf{ok}$, then $\vdash \Gammabar~\mathsf{ok}$.
  \item If $\Gamma \vdash \kappa : \kind$, then $\Gammabar \vdash \kappabar : \kind$.
  \item If $\Gamma \vdash c : \kappa$, then $\Gammabar \vdash \overline{c} : \kappabar$ \ldots
  \item \ldots and so on, until:
    If $\Gamma \vdash e : \tau \leadsto \ebar$, then $\Gammabar \vdash \ebar : \taubar$.
\end{enumerate}

We might want dynamic correctness. But what is it? Dunno.

We might also want coherence: if $\Gamma \vdash e : \tau \leadsto \ebar$ and
$\Gamma \vdash e : \tau \leadsto \ebar'$, then $\Gammabar \vdash \ebar \cong \ebar' : \taubar$
for some suitable $\cong$.
But unfortunately ``$\cong$,'' like the origin of the Donkey Kong Wikia page that claims
that Funky Kong canonically fought in the Great Ape War,
remains shrouded in mystery.


\end{document}
