\sectionwithdate{Introduction to CPS Conversion}{2/6/2018}

First, let's tie up a loose end: structural equivalence of paths.

\begin{judgment}[Structural equivalence]
  $\Gamma \vdash  p \leftrightarrow p' : \kappa$
  \[
    \infer{\Gamma \vdash \alpha \leftrightarrow \alpha : \kappa}
      {\Gamma(\alpha) = \kappa}
  \]
  \[
    \infer[\star]{\Gamma \vdash p~c \leftrightarrow p'~c' : [c/\alpha]\kappa_2}
      {\Gamma \vdash p \leftrightarrow p' : \Pi(\alpha : \kappa_1).\kappa_2
      &\Gamma \vdash c \Leftrightarrow c' : \kappa_1
      }
  \]
  \[
    \infer{\Gamma \vdash \pi_1 p \leftrightarrow \pi_1 p' : \kappa_1}
      {\Gamma \vdash p \leftrightarrow p' : \Sigma(\alpha : \kappa_1).\kappa_2}
    \qquad
    \infer{\Gamma \vdash \pi_2p \leftrightarrow \pi_2p' : [\pi p/\alpha]\kappa_2}
      {\Gamma \vdash p \leftrightarrow p' : \Sigma(\alpha : \kappa_1).\kappa_2}
  \]
  \[ \infer{\Gamma \vdash (c_1 \rightarrow c_2) \leftrightarrow (c_1' \rightarrow c_2') : \T}
      {\Gamma \vdash c_1 \Leftrightarrow c_2' : \T
      &\Gamma \vdash c_2 \Leftrightarrow c_2' : \T
      }
    \qquad
    \infer{\Gamma \vdash \forall(\alpha : \kappa).c \leftrightarrow
            \forall(\alpha : \kappa').c' : \T}
      {\Gamma \vdash \kappa \Leftrightarrow \kappa' : \kind
      &\Gamma, \alpha : \kappa \vdash c \Leftrightarrow c' : \T
      }
  \]
\end{judgment}
In the starred rule, we could have either substituted $c$ or $c'$ for $\alpha$; either
way, the proof is made annoying. We introduced one more judgment along the way:
\begin{judgment}[Kind equivalence]
  $\Gamma \vdash \kappa \Leftrightarrow \kappa' : \kind$
  \[
    \infer{\Gamma \vdash \T \Leftrightarrow \T : \kind}{}
    \qquad
    \infer{\Gamma \vdash \S(c) \Leftrightarrow \S(c') : \kind}
      {\Gamma \vdash c \Leftrightarrow c' : \T}
  \]
  \[
    \infer
      {\deduce
        {\Gamma \vdash \Sigma(\alpha : \kappa_1).\kappa_2 \Leftrightarrow
          \Sigma(\alpha : \kappa_1').\kappa_2' : \kind}
        {\Gamma \vdash \Pi(\alpha : \kappa_1).\kappa_2 \Leftrightarrow
          \Pi(\alpha : \kappa_1').\kappa_2' : \kind}
      }
      {\Gamma \vdash \kappa_1 \Leftrightarrow \kappa_1' : \kind
      &\Gamma, \alpha : \kappa_1 \vdash \kappa_2 \Leftrightarrow \kappa_2' : \kind
      }
  \]
\end{judgment}
Done!

\subsection{CPS conversion}
IL-Direct is roughly the core ML language. In this phase, we will translate IL-Direct
to IL-CPS using, well, CPS conversion.

\paragraph{IL-Direct}
\begin{bnf}
  \tau \bnfeq
  \alpha
  \alt \tau \rightarrow \tau
  \alt \forall(\alpha : \kappa). \tau
  \alt \tau \times \tau
  \alt \exists(\alpha : \kappa).\tau\\
  e \bnfeq x \alt \lambda(x:\tau).e \alt \cdots
\end{bnf}

\paragraph{IL-CPS}
\begin{bnf}
  \kappa \bnfeq \textit{(same as SKC)}\\
  c \bnfeq \textit{(same as SKC)}
  \alt c \times c
  \alt \exists(\alpha : \kappa).c
  \alt \neg c
  \alt \unit\\
  v \bnfeq x \alt \langle v, v \rangle \alt
    \pack{c}{v}{\exists (\alpha : \kappa).\tau}
    \alt \lambda(x : \tau).e \alt \ast\\
  e \bnfeq
    \letv{x}{v}{e}
    \alt \letv{x}{\pi_i v}{e}
    \alt \unpack{\alpha}{x}{v}{e}
    \alt v~v
    \alt \halt
\end{bnf}

Two typing judgments are our main concern at the term level.

\begin{judgment}[Value typing]
  $\Gamma \vdash v : \tau$
  \[
    \infer{\Gamma \vdash x : \tau}{\Gamma(x) = \tau}
    \qquad
    \infer{\Gamma \vdash \langle v_1, v_2 \rangle : \tau_1 \times \tau_2}
      {\Gamma \vdash v_1 : \tau_1
      &\Gamma \vdash v_2 : \tau_2
      }
    \qquad
    \infer{\Gamma \vdash \ast : \unit}{}
  \]
  \[
    \infer{\Gamma \vdash \pack{c}{v}{\exists(\alpha : \kappa).\tau}
      : \exists(\alpha : \kappa).\tau}
      {\Gamma \vdash c : \kappa
      &\Gamma \vdash v : [c/\alpha]\tau
      &\Gamma, \alpha : \kappa \vdash \tau : \T
      }
      \qquad
    \infer{\Gamma \vdash \lambda(x : \tau).e : \neg \tau}
      {\Gamma, x : \tau \vdash e : \mathbf 0
      &\Gamma \vdash \tau : \T
      }
  \]
\end{judgment}

\begin{judgment}[Expression typing]
  $\Gamma \vdash e : \mathbf 0$
  \[
    \infer{\Gamma \vdash \letv x v e : \mathbf{0}}
      {\Gamma \vdash v : \tau
      &\Gamma, x : \tau \vdash e : \mathbf 0
      }
    \qquad
    \infer
      {\Gamma \vdash \letv x {\pi_i v} e: \mathbf 0}
      {\Gamma \vdash v : \tau_1 \times \tau_2
      &\Gamma \vdash x : \tau_I \vdash e : \mathbf 0
      }
  \]
  \[
    \infer
      {\Gamma \vdash \unpack \alpha x v e : \mathbf 0}
      {\Gamma \vdash v : \exists (\alpha : \kappa).\tau
      &\Gamma, \alpha : \kappa, x : \tau \vdash e : \mathbf{0}
      }
  \]
  \[
    \infer
      {\Gamma \vdash v_1~v_2 : \mathbf{0}}
      {\Gamma \vdash v_1 : \neg \tau
      & \Gamma \vdash v_2 : \tau
      }
    \qquad
    \infer{\Gamma \vdash \halt : \mathbf{0}}{}
  \]
\end{judgment}

John Reynolds remarked in his paper \emph{Definitional interpreters for higher-order
programming languages} that CPS conversion resolves any ambiguities in control flow. For example,
it's immediately apparent whether the semantics is call-by-value or call-by-name. We can remark
on additional aspects of CPS conversion. It:
\begin{enumerate}[1.]
  \item names all intermediate computations,
  \item makes control flow explicit, and
  \item reifies continuations.
\end{enumerate}

Many compilers do 1 and 2 in what's known as A-normal form, not to be confused with anormal
form or a normal form. A-normal form is, formally, ``A it until you can't A it anymore.''
A more descriptive name might be monadic form, or two-thirds CPS.

\subsection{Typing derivation--directed translation}
We don't want to do syntax-directed translation, since this would involve putting type
information everywhere in IL-Direct. Here's an example of a type-directed translation
to warm up to IL-CPS.
We define $\overline{\cdot} : \text{IL-$X$ types} \to \text{IL-$Y$ types}$ so that
\begin{align*}
  \text{if}\quad &\tau \quad \text{is an IL-$X$ type,}\\
  \text{then}\quad &\taubar \quad \text{is an IL-$Y$ type.}
\end{align*}

Under this definition, we want to define a judgment
\[ \Gamma \vdash e : \tau \leadsto \ebar \]
such that, if $\Gamma \vdash e : \tau$, then $\Gamma\vdash : \tau \leadsto \ebar$ such that
(roughly) $\Gammabar \vdash \ebar : \taubar$. Just to reiterate: the translation
of $\tau$ to $\taubar$ is syntax-directed; $\ebar$ is just a meta-variable that we will give
meaning to with the translation judgment.

\subsection{Stupidification}
In this section, we define a stupid translation that does nothing meaningful.
\begin{align*}
  \overline{\alpha} &= \alpha &\overline{\T} &= \T\\
  \overline{\tau_1 \to \tau_2} &= \taubar_1 \to \taubar_2 &\overline{\S(c)} &= \S(c)\\
  \overline{\tau_1 \times \tau_2} &= \taubar_1 \times (\taubar_1 \to \taubar_2) &\overline{\Pi(\alpha : \kappa_1).\kappa_2} &= \Pi(\alpha : \kappabar_1).\kappabar_2\\
  &&\overline{\Sigma(\alpha : \kappa_1).\kappa_2} &= \Sigma(\alpha : \kappabar_1).\kappabar_2
\end{align*}
\begin{align*}
  \overline{\varepsilon} &= \varepsilon\\
  \overline{\Gamma, \alpha : \kappa} &= \Gammabar, \alpha : \kappabar\\
  \overline{\Gamma, x : \tau} &= \Gammabar, x : \taubar
\end{align*}

I ellide the rules defining the translation judgment that just recursively translate the
constituent parts. Here are some of the more interesting rules:
\[
  \infer
    {\Gamma \vdash \langle e_1, e_2 \rangle : \tau_1 \times \tau_2
      \leadsto \langle \ebar_1, \lambda(\_ : \taubar_1) : \ebar_2}
    {\Gamma \vdash e_1 : \tau_1 \leadsto \ebar_1
    &\Gamma \vdash e_2 : \tau_2 \leadsto \ebar_2}
  \qquad
  \infer
    {\Gamma \vdash \pi_2e : \tau_2 \leadsto \letv{x}{\ebar}{\pi_2 x(\pi_1 x)}}
    {\Gamma \vdash e : \tau_1 \times \tau_2 \leadsto \ebar}
\]
Now you see why it's important that the translation be type-directed instead of syntax-directed:
otherwise, we wouldn't have known that $\taubar_1$ in the pair introduction rule.

What are some properties we'd like?
\paragraph{(Effectiveness.)}
\begin{enumerate}[1.]
  \item If $c$ is a constructor, then $\overline{c}$ is defined.
    (Likewise for $\kappa$ and $\Gamma$.)
  \item $\Gamma \vdash e : \tau$ is derivable iff $\Gamma \vdash e : \tau \leadsto \ebar$
    for some $\ebar$.
\end{enumerate}
\paragraph{(Static correctness.)}
\begin{enumerate}[1.]
  \item If $\vdash \Gamma~\mathsf{ok}$, then $\vdash \Gammabar~\mathsf{ok}$.
  \item If $\Gamma \vdash \kappa : \kind$, then $\Gammabar \vdash \kappabar : \kind$.
  \item If $\Gamma \vdash c : \kappa$, then $\Gammabar \vdash \overline{c} : \kappabar$ \ldots
  \item \ldots and so on, until:
    If $\Gamma \vdash e : \tau \leadsto \ebar$, then $\Gammabar \vdash \ebar : \taubar$.
\end{enumerate}

We might want dynamic correctness. But what is it? Dunno.

We might also want coherence: if $\Gamma \vdash e : \tau \leadsto \ebar$ and
$\Gamma \vdash e : \tau \leadsto \ebar'$, then $\Gammabar \vdash \ebar \cong \ebar' : \taubar$
for some suitable $\cong$.
But unfortunately ``$\cong$,'' like the origin of the Donkey Kong Wikia page that claims
that Funky Kong canonically fought in the Great Ape War,
remains shrouded in mystery.
