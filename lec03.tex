\sectionwithdate{Explicit substitution}{1/23/2018}

In this lecture, we introduce a representation of substitutions as mathematical objects
rather than resorting to meta-mathematical reasoning. We adopt the presentation of de
Bruijn indices that will be used in the first project.

\subsection{Intuition and Examples}

Throughout, we use an \emph{ordered context}. As a substitution is being performed,
variables are bound to this context. There may still be free variables whose index
exceeds the number of items in the context; we will have to decrement the indices
of these.

Given the term (in de Bruijn indices)
\[ (\lambda. \lambda. 2 + 0 + 3) [M/0]~~~~\text{,} \]
it would be quite nice if this reduced to
\[ (\lambda. \lambda. M + 0 + 2)~~~~\text{.} \]

Notice again how we talk about substitution as being part of the \emph{term} and of
being \emph{reduced}---this is the first sign that we're modeling substitution
explicitly.

The other context explored today is \emph{shifting} (or \text{lifting}), which involves
yet more terms to the ordered context. Let's give some examples of familiar terms and
their de Bruijn equivalents.
\begin{align}
  w &\vdash \lambda y. \lambda z. z+w && \Longrightarrow && \lambda.\lambda. 0 + 2\\
  w, x & \vdash \lambda y. \lambda z. z + w && \Longrightarrow && \lambda. \lambda. 0 + 3
\end{align}
In (2), we must shift the index of $w$ by 1 to refer to the correct position in
the ordered context.

\subsection{Modelling substitutions explicitly}
We model our approach after \url{http://www.hpl.hp.com/techreports/Compaq-DEC/SRC-RR-54.pdf},
taking some liberties, like indexing from 0.

First, our grammar of terms:
\begin{bnf}
  M \bnfeq i \alt \lambda.M \alt M~M \alt M[\sigma]
\end{bnf}
$i$ denotes de Bruijn indices, and postfix brackets indicate substitution. Now, our grammar
of substitutions $\sigma$.
\begin{bnf}
  \sigma \bnfeq M \cdot \sigma \alt \uparrow^n
\end{bnf}
$\cdot$ functions as cons\footnote{In lecture, we used a period ``.'', but this notation is misleadingly
suggestive of binding.}. We usefully abbreviate $\uparrow^0$ as $\id$, and $\uparrow^1$ as
$\uparrow$.

Substitution is given meaning by equations:
\begin{align*}
  0[M \cdot \sigma] &= M\\
  (i+1)[M \cdot \sigma] &= i[\sigma]\\
  i[\uparrow^n] &= i+n\\
  (M_1~M_2)[\sigma] &= M_1[\sigma]~M_2[\sigma]\\
  (\lambda.M)[\sigma] &= \lambda.M[0 \cdot (\sigma \circ \uparrow)]
\end{align*}

Unlike the reference, we don't model $\circ$ as primitive but rather define it as a binary operator
over substitutions. Our goal is for $M[\sigma \circ \sigma'] = M[\sigma][\sigma']$.
Here's how you compute it, assuming $\cdot$ to bind tighter than $\circ$.
\begin{align*}
  \uparrow^m &\circ \uparrow^n &&= \uparrow^{m+n}\\
  \id &\circ M \cdot \sigma &&= M \cdot \sigma\\
  \uparrow^{m+1} &\circ M \cdot \sigma &&= \uparrow^n \cdot \sigma\\
  M \cdot \sigma &\circ \sigma' &&= M[\sigma'] \cdot (\sigma \circ \sigma')
\end{align*}
Composition here can be pronounced ``before.'' The last rule is the only interesting one, indicating
that later substitutions may depend on prior ones, since we must perform $M[\sigma']$.

For example, we would like the following to hold:
\begin{align*}
  (0+1) [M \cdot \uparrow \circ \uparrow^4] &= (0+1) [M \cdot \uparrow][\uparrow^4]\\
  &= (M + 1)[\uparrow^4]\\
  &= M[\uparrow^4] + 5
\end{align*}
And by our equations, it in fact does.
\begin{align*}
  (0 + 1) [M \cdot \uparrow \circ \uparrow^4] &= (0 + 1) [M[\uparrow^4] \cdot (\uparrow \circ \uparrow^4)]\\
  &= (0 + 1) [M[\uparrow^4] \cdot \uparrow^5]\\
  &= M[\uparrow^4] + 5
\end{align*}

\subsection{Rule conversions}
With these notions of indices and binding, we convert some rules from
from \thref{weak} and \thref{algequiv}. Importantly, the context becomes ordered with this conversion.

\begin{judgment}[Rule conversions with de Bruijn indices]
\[
  \vcenter{
    \infer{(\lambda (\alpha : \kappa). c)~c' \leadsto [c'/\alpha] c}{}
  }
  \qquad
  \Longrightarrow_\emph{de Bruijn}
  \qquad
  \vcenter{
    \infer{(\lambda \kappa.c)~c' \leadsto c[ c' \cdot \id]}{}
  }
\]
\[
  \vcenter{
    \infer
      {\Gamma \vdash c \Leftrightarrow c' : \kappa_1 \rightarrow \kappa_2}
      {\Gamma, \alpha : \kappa_1 \vdash c~\alpha \Leftrightarrow c'~\alpha : \kappa_2}
  }
  \qquad
  \Longrightarrow_\emph{de Bruijn}
  \qquad
  \vcenter{
    \infer
      {\Gamma \vdash c \Leftrightarrow c' : \kappa_1 \rightarrow \kappa_2}
      {\Gamma, \kappa_1 \vdash c[\uparrow]~0 \Leftrightarrow c'[\uparrow]~0 : \kappa_2}
  }
\]
\end{judgment}

\subsection{Substitutions we'll use}
Rather than using substitutions in their full generality, we are really concerned with substitutions
of the form $[ 0 \cdot 1 \cdots n-1 \cdot M_1[\uparrow^n] \cdots M_k[\uparrow^n] \cdot \uparrow^{n+\ell}]$.
This is still pretty general:
\begin{enumerate}[1.]
  \item $[M \cdot \id]$ has $n = 0, k = 1, \ell = 0$.
  \item $[ \uparrow^\ell ]$ has $n = 0, k = 0$.
  \item For $\sigma$ of the desired form, the substitution $[0 \cdot (\sigma \circ \uparrow)]$ retains the form:
    $$0 \cdot (( 0 \cdot 1 \cdots n-1 \cdot M_1[\uparrow^n] \cdots M_k[\uparrow^n] \cdot \uparrow^{n+\ell}) \circ \uparrow)$$
    $$\Downarrow$$
    $$0 \cdot 1 \cdots n \cdot M_1[\uparrow^{n+1}] \cdots M_k[\uparrow^{n+1}] \cdot \uparrow^{n+\ell+1}$$

\subsection{The type theory of explicit substitution}
Assuming the type theory of the underlying language, we want to show that:
\[  \textit{If} \quad \Gamma \vdash \sigma : \Gamma'
     \quad \textit{and} \quad \Gamma' \vdash M : A, \quad \textit{then} \quad \Gamma \vdash M[\sigma] : A[\sigma].
\]

There are a few things to note. One is that the type of a substitution $\sigma$ can be described be
an ordered list of types; in other words, the type of $\sigma$ is a context $\Gamma'$. Next,
the use of $A[\sigma]$ indicates that the type $A$ may refer to variables in the context $\Gamma$.
This will require some thought about dependent types. Finally, this framework applies
equally well to $M$ indicating a term and $A$ a type as it does to $M$ indicating a type
and $A$ a kind.

Below, we assume $\Gamma$ to be an ordered context of the form $\varepsilon$ or $\Gamma, A$.
\begin{judgment}[Substitution typing---general rules]
\[
  \infer{\Gamma \vdash M \cdot \sigma : \Gamma', A}
    {\Gamma \vdash \sigma : \Gamma' & \Gamma \vdash M : A[\sigma]}
\]
\[
  \infer{\Gamma \vdash \uparrow^k : \Gamma'}
    {\Gamma = \Gamma', A_1, \ldots, A_k}
  \qquad
  \infer{\varepsilon \vdash \id : \varepsilon}{}
\]
\end{judgment}

We can use these rules to perform a simple derivation after converting to de Bruijn indices.
We wish to derive the RHS of:
\[
  y : \int \vdash [\int/\alpha, y/x] : \alpha : \T, x : \alpha
    \qquad \Longrightarrow_\textnormal{deBruijn} \qquad
    \int \vdash \int \cdot 0 \cdot \uparrow : \varepsilon, \T, 0
\]
We show this, explicitly including $\varepsilon$ as part of the types of substitutions
(so that it is slightly easier to distinguish judgments on the types of substitutions
and judgments on the kinds of types).
\[
  \infer
    {\int \vdash 0 \cdot \int \cdot \uparrow : \varepsilon, \T, 0}
    {\infer
       {\int \vdash \int \cdot \uparrow : \varepsilon, \T}
       {\infer
          {\int \vdash \uparrow : \varepsilon}
          {}
       &\infer
          {\int \vdash \int : \T[\uparrow]}
          {}
       }
    &\infer
       {\int \vdash 0 : 0[\int.\uparrow]}
       {\infer
          {\int \vdash 0 : \int}
          {}
       }
    }
\]

We perform transformations on the rules of \thref{kind}, now, in this type-theoretic setting.
\begin{judgment}[Type kind with substitutions]
  \[
    \vcenter{ \infer{\Gamma \vdash \alpha : \kappa}{\Gamma(\alpha) : \kappa} }
    \qquad
    \Longrightarrow
    \qquad
    \vcenter{ \infer{\Gamma \vdash i : \kappa[\uparrow^{i+1}]}{\Gamma(i) : \kappa} }
  \]
  \[
    \vcenter{
      \infer{\Gamma \vdash \lambda(\alpha : \kappa).c : \kappa \to \kappa'}
            {\Gamma, \alpha : \kappa \vdash c : \kappa'}
    }
    \qquad
    \Longrightarrow
    \qquad
    \vcenter{
      \infer{\Gamma \vdash \lambda \kappa. c : \kappa \rightarrow \kappa'}
           {\Gamma, \kappa \vdash c : \kappa'[\uparrow]}
    }
  \]
  Unchanged rules:
  \[
  \infer{\Gamma \vdash c_1 ~ c_2 : \kappa'}
        {\Gamma \vdash c_1 : \kappa \rightarrow \kappa'
        &\Gamma \vdash c_2 : \kappa
        }
  \qquad
  \infer{\Gamma \vdash \forall(\alpha : \kappa).c : \T}
        {\Gamma, \alpha : \kappa \vdash c : \T}
  \]
\end{judgment}
The $\forall$ rule may be surprising, since there are no shifts involved, but this is only
because there's no need to shift $\T$ up as in the lambda rule; $\T$ has no variables.

Finally, we give the same sort of transformations for our term language, transforming
rules from \thref{term}.
\begin{judgment}[Term type with substitution]
  \[
    \vcenter{ \infer{\Gamma \vdash x : \tau}{\Gamma(x) : \tau} }
    \qquad
    \Longrightarrow
    \qquad
    \vcenter{ \infer{\Gamma \vdash i : \tau[\uparrow^{i+1}]}{\Gamma(i) : \tau} }
  \]
  \[
    \vcenter{
      \infer{\Gamma \vdash e[\tau] : [\tau / \alpha]\tau'}
          {\Gamma \vdash e : \forall(\alpha : \kappa).\tau'
          &\Gamma \vdash \tau : \kappa
          }
    }
    \qquad
    \Longrightarrow
    \qquad
    \vcenter{
      \infer{\Gamma \vdash e[c \cdot \id] : \tau [c \cdot \id]}
        {\Gamma \vdash e : \forall \kappa. \tau
        & \Gamma \vdash c : \kappa
        }
    }
  \]
  \[
    \vcenter{
      \infer{\Gamma \vdash \lambda(x : \tau).e : \tau \rightarrow \tau'}
            {\Gamma, x : \tau \vdash e : \tau'
            &\Gamma \vdash \tau : \kappa
            }
    }
    \qquad
    \Longrightarrow
    \qquad
    \vcenter{
      \infer{\Gamma \vdash \lambda \tau. e : \tau \rightarrow \tau'}
           {\Gamma, \tau \vdash e : \tau'[\uparrow]}
    }
  \]
  Unchanged rules:
  \[
    \infer{\Gamma \vdash \Lambda (\alpha : \kappa).e : \forall (\alpha : \kappa).\tau}
          {\Gamma, \alpha : \kappa \vdash e : \tau}
    \qquad
    \infer{\Gamma \vdash e_1 ~ e_2 : \tau'}
          {\Gamma \vdash e_1 : \tau \rightarrow \tau'
          &\Gamma \vdash e_2 : \tau
          }
  \]
\end{judgment}

Note that the lambda rule above was extrapolated by myself and not specifically covered
in lecture. Mistakes are mine. It is known.

\end{enumerate}
