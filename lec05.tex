\sectionwithdate{Close Encounters of the Singleton Kind}{1/30/2018}

As an introduction to this lecture, we extend \thref{sgk:type} with more rules.
\begin{judgment}[Constructor formation (still incomplete)]\thlabel{sgk:incomplete}
  $\Gamma \vdash c : \kappa$
  \[
    \infer{\Gamma \vdash \alpha : \kappa}{\Gamma(\alpha) = \kappa}
  \]
  \[
    \infer{\Gamma \vdash \lambda(\alpha : \kappa).c : \Pi(\alpha : \kappa).\kappa'}
      {\Gamma \vdash \kappa : \kind
      &\Gamma, \alpha : \kappa \vdash c : \kappa'
      }
    \qquad
    \infer{\Gamma \vdash c_1~c_2 : [c_2 / \alpha]\kappa'}
      {\Gamma \vdash c_1 : \Pi(\alpha:\kappa).\kappa'
      &\Gamma \vdash c_2 : \kappa
      }
  \]
  \[
    \infer{\Gamma \vdash \langle c_1, c_2 \rangle : \Sigma(\alpha : \kappa_1).\kappa_2}
      {\Gamma, \alpha : \kappa_1 \vdash \kappa_2 : \kind
      &\Gamma \vdash c_1 : \kappa_1
      &\Gamma \vdash c_2 : [c_1/\alpha]\kappa_2
      }
    \qquad
  \]
  \[
    \infer{\Gamma \vdash \pi_1c : \kappa_1}
      {\Gamma \vdash c : \Sigma (\alpha : \kappa_1).\kappa_2}
    \qquad
    \infer{\Gamma \vdash \pi_2c : [\pi_1c/\alpha]\kappa_2}
      {\Gamma \vdash c : \Sigma (\alpha : \kappa_1).\kappa_2}
  \]
Same as before:
  \[
    \infer{\Gamma \vdash c_1 \to c_2 : \T}{\Gamma \vdash c_1 : \T & \Gamma \vdash c_2 : \T}
    \qquad
    \infer{\Gamma \vdash \forall(\alpha:\kappa).c : \T}
      {\Gamma \vdash \kappa : \kind
      &\Gamma, \alpha : \kappa \vdash c : \T
      }
    \qquad
    \infer{\Gamma \vdash c : \S(c)}
      {\Gamma \vdash c : \T}
  \]
\end{judgment}
We will add more rules to this as we go.

\subsection{Higher-order singletons}
With no effort, we can say $\int : \S(\int)$. However, we have no notion
of $\lst : \S(\lst)$, even though we can encode this as the judgment
$\lst : \Pi(\alpha : \T). \S(\lst~\alpha)$, where the kind on the RHS
uniquely determines the type.

\paragraph{Higher-order singleton. (Defined notion.)}
\begin{align*}
  \S(c : \T) &= \S(c)\\
  \S(c : \Pi(\alpha : \kappa_1).\kappa_2) &= \Pi(\alpha : \kappa_1).\S(c~\alpha : \kappa_2)\\
  \S(c : \Sigma(\alpha : \kappa_1).\kappa_2) &=
    \S(\pi_1c : \kappa_1) \times \S(\pi_2c : [\pi_1c/\alpha]\kappa_2)\\
  \S(c : \S(c')) &= \S(c)
\end{align*}
It is for ease of proof that we write the third rule non-dependently and the
fourth rule as resolving to $c$ instead of $c'$. The third rule might be nicer in a
de Bruijn setting, since there is no shifting involved with it.

You might like the rules \thref{sgk:incomplete} in their current form, but
these rules aren't sufficient to prove seemingly-trivial theorems. For example,

\[ \infer{\lst : \Pi(\alpha : \T). \S(\lst~\alpha)}
    {\lst : \T \to \T
    &\deduce{\T \to \T \le \Pi(\alpha : \T) . \S(\lst~\alpha)}{\textcolor{red}{\times}}
    }
\]
Therefore, we need something like $\eta$-expansion, which we have by convention
called extensionality rules. In addition to \thref{sgk:incomplete}, we add the rules:
\begin{judgment}[Constructor formation (extensionality)]\thlabel{sgk:extensionality}
  \[
    \infer{\Gamma \vdash c : \Pi(\alpha : \kappa_1).\kappa_2}
      {\Gamma, \alpha : \kappa_1 \vdash c~\alpha : \kappa_2
      &\Gamma \vdash \kappa_1 : \kind
      }
    \qquad
    \infer{\Gamma \vdash c : \Sigma(\alpha : \kappa_1).\kappa_2}
      {\Gamma \vdash \pi_1c : \kappa_1
      &\Gamma \vdash \pi_2c : [\pi_1c/\alpha]\kappa_2
      &\Gamma, \alpha : \kappa_1 \vdash \kappa_2 : \kind
      }
  \]
\end{judgment}

We need these rules to show that:
\begin{align*}
  \text{If} \quad \Gamma \vdash c : \kappa \quad \text{and} \quad &\vdash \Gamma~\mathsf{ok},\\
  \text{then} \quad \Gamma &\vdash \S(c : \kappa) : \kind\\
  \text{and} \quad \Gamma &\vdash c : \S(c : \kappa)
\end{align*}

\begin{judgment}[Kind equivalence (boring)]
  \[
    \infer{\Gamma \vdash \kappa \equiv \kappa : \kind}{\Gamma \vdash \kappa : \kind}
    \qquad
    \infer{\Gamma \vdash \kappa \equiv \kappa' : \kind}
          {\Gamma \vdash \kappa' \equiv \kappa : \kind}
    \qquad
    \infer{\Gamma \vdash \kappa_1 \equiv \kappa_3 : \kind}
          {\Gamma \vdash \kappa_1 \equiv \kappa_2 : \kind
          &\Gamma \vdash \kappa_2 \equiv \kappa_3 : \kind
          }
  \]
  \[
    \infer{\Gamma \vdash \S(c) \equiv \S(c') : \kind}
          {\Gamma \vdash c \equiv c' : \T}
    \qquad
    \infer{
      \deduce
        {\Gamma \vdash \Pi(\alpha : \kappa_1).\kappa_2 \equiv
          \Pi(\alpha : \kappa_1').\kappa_2' : \kind}
        {\Gamma \vdash \Sigma(\alpha : \kappa_1).\kappa_2 \equiv
          \Sigma(\alpha : \kappa_1').\kappa_2' : \kind}}
      {\Gamma \vdash \kappa_1 \equiv \kappa_1' : \kind
      &\Gamma, \alpha : \kappa_1 \vdash \kappa_2 \equiv \kappa_2' : \kind
      }
  \]
\end{judgment}

\begin{judgment}[Subkinding]\mbox{}\\
Preorder:
  \[
    \infer{\Gamma \vdash \kappa \le \kappa'}
      {\Gamma \vdash \kappa \equiv \kappa' : \kind}
    \qquad
    \infer{\Gamma \vdash \kappa_1 \le \kappa_3}
      {\Gamma \vdash \kappa_1 \le \kappa_2
      &\Gamma \vdash \kappa_2 \le \kappa_3
      }
  \]
Other:
  \[
    \infer{\Gamma \vdash \S(c) \le \T}{\Gamma \vdash c : \T}
  \]
  \[
    \infer
      {\Gamma \vdash \Pi(\alpha : \kappa_1).\kappa_2 \le \Pi(\alpha : \kappa_1') . \kappa_2'}
      {\Gamma \vdash \kappa_1' \le \kappa_1
      &\Gamma, \alpha : \kappa_1' \vdash \kappa_2 \le \kappa_2'
      &\Gamma, \alpha : \kappa_1 \vdash \kappa_2 : \kind
      }
  \]
  \[
    \infer
      {\Gamma \vdash \Sigma(\alpha : \kappa_1).\kappa_2 \le \Sigma(\alpha : \kappa_1').\kappa_2'}
      {\Gamma \vdash \kappa_1 \le \kappa_1'
      &\Gamma, \alpha : \kappa_1 \vdash \kappa_2 \le \kappa_2'
      &\Gamma, \alpha : \kappa_1' \vdash \kappa_2' : \kind
      }
  \]
\end{judgment}
The rule for $\Pi$ reminds us that functions are contravariant in the domain and
covariant in the codomain. Notice that we choose the more specific $\alpha : \kappa_1'$
when checking $\kappa_2 \le \kappa_2'$; this is because the wider $\kappa_1$ would not make sense
in $\kappa_2$. But that's why we still must check the validity of $\alpha : \kappa_1$
in $\kappa_2$.

From these subsumption rules, we can derive a few useful results:
\[
  \infer
    {\Gamma \vdash \S(c) \le \S(c')}
    {\Gamma \vdash c \equiv c' : \T}
  \qquad
  \infer
    {\Gamma \vdash c : \S(c')}
    {\Gamma \vdash c \equiv c' : \T}
\]

\begin{judgment}[Constructor equivalence]\mbox{}\\
Equivalence relation:
  \[
    \infer{\Gamma \vdash c \equiv c : \kappa}{\Gamma \vdash c : \kappa}
    \qquad
    \infer{\Gamma \vdash c' \equiv c : \kappa}{\Gamma \vdash c \equiv c'}
    \qquad
    \infer{\Gamma \vdash c_1 \equiv c_3 : \kappa}
      {\Gamma \vdash c_1 \equiv c_2 : \kappa
      &\Gamma \vdash c_2 \equiv c_3 : \kappa
      }
  \]
Compatibility rules:
  \[
    \infer{\Gamma \vdash \lambda(\alpha : \kappa_1).c \equiv \lambda(\alpha : \kappa_1').c'
      : \Pi(\alpha : \kappa_1).\kappa_2}
      {\Gamma \vdash \kappa_1 \equiv \kappa_1' : \kind
      &\Gamma, \alpha : \kappa_1 \vdash c \equiv c' : \kappa_2
      }
    \qquad
    \infer{\Gamma \vdash c_1~c_2 \equiv c_1'~c_2' : [c_2/\alpha]\kappa_2}
      {\Gamma \vdash c_1 \equiv c_1' : \Pi(\alpha : \kappa_1).\kappa_2
      &\Gamma \vdash c_2 \equiv c_2' : \kappa_2
      }
  \]
  \[
    \infer{\Gamma \vdash \langle c_1, c_2 \rangle \equiv \langle c_1', c_2' \rangle
      : \Sigma(\alpha : \kappa_1).\kappa_2}
      {\Gamma \vdash c_1 \equiv c_1' : \kappa_1
      &\Gamma \vdash c_2 \equiv c_2' : [c_1 / \alpha]\kappa_2
      }
    \qquad
    \infer{
      \deduce
         {\Gamma \vdash \pi_2c \equiv \pi_2 c' : [\pi_1c/\alpha]\kappa_2}
         {\Gamma \vdash \pi_1c \equiv \pi_1 c' : \kappa_1}
      }
      {\Gamma \vdash c \equiv c' : \Sigma(\alpha : \kappa_1).\kappa_2}
  \]
  Type constructor rules:
  \[
    \infer[(\beta)]
      {\Gamma \vdash (\lambda (\alpha : \kappa_1). c_2)~c_1 \equiv
        [c_1/\alpha]c_2 : [c_1/\alpha]\kappa_2}
      {\Gamma \vdash c_1 : \kappa_1
      &\Gamma, \alpha : \kappa_1 \vdash c_2 : \kappa_2
      }
  \]
  \[
    \infer{\Gamma \vdash \pi_1\langle c_1, c_2 \rangle \equiv c_1 : \kappa_1}
      {\Gamma \vdash c_1 : \kappa_1
      &\Gamma \vdash c_2 : \kappa_2
      }
    \qquad
    \infer{\Gamma \vdash \pi_2\langle c_1, c_2 \rangle \equiv c_2 : \kappa_2}
      {\Gamma \vdash c_1 : \kappa_1
      &\Gamma \vdash c_2 : \kappa_2
      }
  \]
  \[
    \infer{\Gamma \vdash c_1 \to c_2 \equiv c_1' \to c_2' : \T}
      {\Gamma \vdash c_1 \equiv c_1' : \T
      &\Gamma \vdash c_2 \equiv c_2' : \T
      }
    \qquad
    \infer{\Gamma \vdash \forall (\alpha : \kappa).c \equiv \forall(\alpha : \kappa').c' : \T}
      {\Gamma \vdash \kappa \equiv \kappa' : \kind
      &\Gamma, \alpha : \kappa \vdash c \equiv c' : \T
      }
  \]
Extensionality rules:
\[
  \infer{\Gamma \vdash c \equiv c' : \Pi(\alpha : \kappa_1) \kappa_2}
    {\Gamma \vdash \kappa_1 : \kind
    &\Gamma, \alpha : \kappa_1 \vdash c~\alpha \equiv c'~\alpha
    &\deduce{\left[ \Gamma \vdash c \equiv c' : \Pi(\alpha : \kappa_1).\kappa_2' \right]}
      {
        \left[\deduce{\Gamma \vdash c : \Pi(\alpha : \kappa_1).\kappa_2'}
              {\Gamma \vdash c' : \Pi(\alpha : \kappa_1).\kappa_2''}\right]
      }
    }
\]
\[
  \infer{\Gamma \vdash c \equiv c' : \Sigma(\alpha : \kappa_1).\kappa_2}
    {\Gamma \vdash \pi_1c \equiv \pi_1c' : \kappa_1
    &\Gamma \vdash \pi_2c \equiv \pi_2c' : [\pi_1c/\alpha]\kappa_2
    &\Gamma, \alpha : \kappa_1 \vdash \kappa_2 : \kind
    }
\]
\end{judgment}
The $\forall$ type constructor rule is the only reason we had to define kind equivalence; for the
other rules where we invoke equivalence, we could have simply used structural equality of kinds.
The bracketed premises in the $\Pi$ extensionality rule are the so-called ``nuisance premises''
needed for the proofs to go through. There are two sets of bracketed premises, so there are
actually two rules differing only in which set of bracketed premises you choose.\\

\noindent\emph{Singletons:}
\[
  \infer{\Gamma \vdash c \equiv c' : \kappa'}
    {\Gamma \vdash c \equiv c' : \kappa
    &\Gamma \vdash \kappa \le \kappa'
    }
  \qquad
  \infer[\star]{\Gamma \vdash c \equiv c' : \T}{\Gamma \vdash c : \S(c')}
  \qquad
  \infer{\Gamma \vdash c \equiv c' : \S(c)}{\Gamma \vdash c \equiv c' : \T}
\]

Phew!
For term formation, only one rule changes: checking kind ok-ness for
type abstraction.

(We went through some examples in lecture, but nothing that doesn't follow from the rules
given above. Maybe I'll add them later.)

%TODO: Add examples
