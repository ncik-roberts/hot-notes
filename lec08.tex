\sectionwithdate{CPL in CPS}{2/8/2018}

The main judgment for today is:
\[ \Gamma \vdash e : \tau \leadsto k . \ebar \]

Ideally, if $\Gamma \vdash e : \tau \leadsto k.\ebar$, then
$\Gammabar, k : \neg \taubar \vdash \ebar$. You might be ``k''onfused as to why we
use $k$ to stand for continuation. Well, it's the law.

\begin{judgment}[Type translation for IL-CPS]
  \begin{align*}
    \overline{\alpha} &= \alpha\\
    \overline{\tau_1 \times \tau_2} &= \taubar_1 \times \taubar_2\\
    \overline{\tau_1 \to \tau_2} &= \neg (\taubar_1 \times \neg \taubar_2)\\
    \overline{\exists(\alpha : \kappa).\tau} &= \exists (\alpha : \kappabar).\taubar\\
    \overline{\forall(\alpha : \kappa).\tau} &= \neg(\exists (\alpha : \kappabar).\neg\taubar)\\
  \end{align*}
\end{judgment}

\begin{judgment}[Kontinuation konversion]
  $\Gamma \vdash e : \tau \leadsto k . \ebar$
  \[
    \infer
      {\Gamma \vdash x : \tau \leadsto k.k~x}
      {\Gamma(x) = \tau}
    \qquad
    \qquad
    \textnormal{Check:}
    \quad
    \infer
      {\Gammabar, k : \neg \taubar \vdash k~x : \mathbf{0}}
      {\Gammabar, k : \neg \taubar \vdash k : \neg \taubar
      &\Gammabar \vdash x : \taubar
      }
  \]

  \[
    \infer
      {\deduceinv
        {\Gamma \vdash \langle e_1, e_2 \rangle : \tau_1 \times \tau_2 \leadsto}
        {\deduceinv
          {k. \qqqquad}
          {\deduceinv
            {\qqquad \mathtt{let}~k_1 = \big(\lambda(x_1 : \taubar_1).}
            {\deduceinv
              {\qqqqqquad
                \mathtt{let}~k_2 = \big(\lambda(x_2 : \taubar_2). k\langle x_1, x_2 \rangle\big)}
              {\deduceinv
                {\mathtt{in}~\ebar_2 \quad}
                {\big) \mathtt{in}~\ebar_1 \qquad}}}}}}
      {\Gamma \vdash e_1 : \tau_1 \leadsto k_1.\ebar_1
      &\Gamma \vdash e_2 : \tau_2 \leadsto k_2.\ebar_2
      }
  \]

 \[
   \infer
     {\deduceinv
       {\Gamma \vdash \pi_ie : \tau_i \leadsto}
       {\deduceinv
         {k. \qqquad}
         {\deduceinv
           {\qqqqquad \mathtt{let}~k' = \big(\lambda(x : \taubar_1 \times \taubar_2).}
           {\deduceinv
             {\qqqqquad \letv{y}{\pi_i x}{k~y}}
             {\big) \mathtt{in}~\ebar}}}}}
    {\Gamma \vdash e : \tau_1 \times \tau_2 \leadsto k'.\ebar}
 \]

 \[
   \infer
     {\deduceinv
       {\Gamma \vdash \lambda(x : \tau).e : \tau \to \tau' \leadsto}
       {\deduceinv
         {k.\qqqqquad}
         {\deduceinv
           {k(\lambda (y : \taubar \times \neg \taubar').}
           {\deduceinv
             {\qqquad \mathtt{let}~x = \pi_1 y~\mathtt{in}}
             {\deduceinv
               {\qqquad \mathtt{let}~k' = \pi_2 y~\mathtt{in}}
               {\ebar\big) \quad}}}}}}
    {\Gamma \vdash \tau : \T
    &\Gamma, x : \tau \vdash e : \tau' \leadsto k'.\ebar}
 \]

 \[
   \infer
     {\deduceinv
       {\Gamma \vdash e_1~e_2 : \tau' \leadsto}
       {\deduceinv
         {k. \qqqquad}
         {\deduceinv
           {\qqqqquad \mathtt{let}~k_1 = \big(\lambda (f : \neg(\taubar \times \neg \taubar')).}
           {\deduceinv
             {\qqqqqquad \mathtt{let}~k_2 = \big( \lambda(x : \taubar). f\langle x, k \rangle \big)}
             {\deduceinv
               {\quad \mathtt{in}~\ebar_2}
               {\big) \mathtt{in}~\ebar_1 \quad}}}}}}
    {\Gamma \vdash e_1 : \tau \to \tau' \leadsto k_1.\ebar_1
    &\Gamma \vdash e_2 : \tau \leadsto k_2. \ebar_2
    }
 \]

 %TODO: format
 \[
   \infer{\Gamma \vdash \pack{c}{e}{\exists(\alpha : \kappa).\tau}
     : \exists(\alpha : \kappa).\tau
     \leadsto k. \letv{k'}{\lambda(x : \overline{[c/\alpha]\tau})
     k(\pack{\overline{c}}{x}{\exists(\alpha : \kappabar).\taubar})}{\ebar}}
    {\Gamma \vdash c : \kappa
    &\Gamma \vdash e : [c/\alpha] \tau
    &\Gamma, \alpha : \kappa \vdash \tau : \T
    }
 \]

 \[
   \infer
     {\Gamma \vdash \unpack{\alpha}{x}{e_1}{e_2} : \tau_2
       \leadsto k. \letv{k_1}{(\lambda(y : \exists(\alpha : \kappabar).\taubar).
         \unpack{\alpha}{x}{y}{\letv{k_2}{k}{\ebar_2}})}{\ebar_1}}
     {\Gamma \vdash e_1 : \exists(\alpha : \kappa).\tau
     &\Gamma, \alpha : \kappa, x : \tau_1 \vdash e_2 : \tau_2 \leadsto \kappa_2.\ebar_2
     &\Gamma \vdash \tau_2 : \T
     }
 \]

 \[
   \infer{\Gamma \vdash \Lambda(\alpha : \kappa).e : \forall(\alpha : \kappa).\tau \leadsto
     k. k\big(\lambda(x : \exists(\alpha : \kappabar).\neg \taubar).
     \unpack{\alpha}{k'}{x}{\ebar}}
     {\Gamma \vdash k : \kind
     &\Gamma, \alpha : \kappa \vdash e : \tau \leadsto k' . \ebar
     }\big)
\]
\end{judgment}

To show that the $\mathtt{pack}$/$\mathtt{unpack}$ rules are sound, we would need
a lemma that:
\[ \overline{[c_1/\alpha]c_2} = [\overline{c}_1 / \alpha]\overline{c}_2 \]
This works provided that $\overline{\alpha} = \alpha$. Phrased differently,
``substitution commutes with translation.'' Finally, the judgment $\Gamma \vdash \tau_2 : \T$
is needed in the premise of the $\mathtt{unpack}$ rule so that $\alpha$ doesn't escape
its scope.
